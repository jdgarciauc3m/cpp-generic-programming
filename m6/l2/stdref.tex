\subsection{std::ref() and std::cref()}

\begin{frame}[t,fragile]{Helper functions}
\begin{itemize}
  \item The easiest way to create a \cppid{reference\_wrapper<T>} from l-values 
        is through helper functions:
    \begin{itemize}
      \item \cppid{std::ref(x)}: A \cppid{reference\_wrapper<T>} for \cppid{x}.
      \item \cppid{std::cref(x)}: A \cppid{reference\_wrapper<T const>} for \cppid{x}.
    \end{itemize}
\end{itemize}

\mode<presentation>{\vspace{-.75em}}
\begin{columns}[T]

\column{.5\textwidth}
\begin{block}{A generic call interface}
\begin{lstlisting}[basicstyle=\mode<presentation>{\tiny}]
void update(int &x) {
  ++x;
}

template<typename F, typename T>
void call(F fun, T arg) {
  std::cout << "log: Calling function\n";
  fun(arg);
  std::cout << "log: Function executed\n";
}
\end{lstlisting}
\end{block}

\pause
\column{.5\textwidth}
\begin{block}{Calling the function}
\begin{lstlisting}[basicstyle=\mode<presentation>{\tiny},escapechar=@]
void f() {
  int val = 42;
  call(update, @\color{red}std::ref@(val));
  std::cout << std::format("val = {}\n", val);
}
\end{lstlisting}
\end{block}

\pause
\begin{lstlisting}[style=terminal,basicstyle=\mode<presentation>{\tiny}]]
log: Calling function
log: Function executed
val = 43
\end{lstlisting}

\end{columns}
\end{frame}


\begin{frame}[t,fragile]{Binding arguments to callables}
\begin{itemize}
  \item \cppid{std::bind()} generates a function object that can be invoked later.

    \mode<presentation>{\vfill\pause}
    \begin{itemize}
      \item From an invocable object and given concrete parameters.
\begin{lstlisting}
void print(int x, int y, int z);
auto g = std::bind(print, 1, 3, 5);
g(); // Invokes print(1,3,5)
\end{lstlisting}

      \mode<presentation>{\vfill\pause}
      \item May use some binding placeholder.
\begin{lstlisting}
void print(int x, int y, int z);
using namespace std::place_holders
auto g = std::bind(print, 100, _1, _2);
g(42,99); // Invokes print(100,42,99)
g(23,35); // Invokes print(100,23,35)
\end{lstlisting}
    \end{itemize}
\end{itemize}
\end{frame}

\begin{frame}[t,fragile]{Binding and reference parameters}
\begin{itemize}
  \item \textbad{Problem}: When not using a placeholder \cppid{std::bind()} makes a copy
        of the passed argument.
    \begin{itemize}
      \item Problematic for l-values.
    \end{itemize}
\end{itemize}

\begin{block}{Problems with std::bind()}
\begin{lstlisting}[basicstyle=\mode<presentation>{\tiny}]
void update(double & value, double delta) {
  value += delta;
}

void void f() {
  double x = 1.2; 
  auto g = std::bind(update, _1, .5);
  g(x);
  std::cout << std::format("x = {}\n", x); // Prints 1.7

  auto h = std::bind(update,x, .25);
  h();
  std::cout << std::format("x = {}\n", x); // Ouch. Prints 1.7
}
\end{lstlisting}
\end{block}
\end{frame}

\begin{frame}[t,fragile]{Correctly binding reference parameters}
\begin{itemize}
  \item Use \cppid{std::ref()} and \cppid{std::cref()}.
\end{itemize}

\begin{block}{Solving problems with std::bind()}
\begin{lstlisting}[basicstyle=\mode<presentation>{\tiny}]
void update(double & value, double delta) {
  value += delta;
}

void void f() {
  double x = 1.2; 
  auto g = std::bind(update, _1, .5);
  g(x);
  std::cout << std::format("x = {}\n", x); // Prints 1.7

  auto h = std::bind(update, std::ref(x), .25);
  h();
  std::cout << std::format("x = {}\n", x); // Prints 1.95
}
\end{lstlisting}
\end{block}
\end{frame}

\begin{frame}[t,fragile]{The good and the bad}
\begin{itemize}
  \item \textbad{Avoid} using \cppid{std::bind()} $\Rightarrow$ Use lambdas instead.
\begin{block}{Solving problems with lambdas}
\begin{lstlisting}[basicstyle=\mode<presentation>{\tiny}]
void update(double & value, double delta) {
  value += delta;
}

void void f() {
  double x = 1.2; 
  auto g = [](double & val) { update(val, 0.5); };
  g(x);

  auto h = [&x]() { update(x, 0.5); };
  h();
}
\end{lstlisting}
\end{block}

  \pause
  \item \textemph{Remember} that you will still need \cppid{std::ref()} and \cppid{std::cref()}
        for \cppid{std::thread}.

\end{itemize}
\end{frame}
