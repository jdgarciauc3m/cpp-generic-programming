\subsection{Syntax of variadic functions}

\begin{frame}[t,fragile]{Defining a variadic function template}
\begin{itemize}
  \item A \textgood{template parameter} may be \textmark{variadic}.
    \begin{itemize}
      \item Use ellipsis before template parameter name.
      \item The parameter maps to a list of types.
    \end{itemize}

  \mode<presentation>{\vfill}
  \item A \textgood{function parameter} may be \textmark{variadic}.
    \begin{itemize}
      \item Use ellipsis between variadic type name and parameter name.
      \item The parameter maps to a list of values of different types.
    \end{itemize}

\mode<presentation>{\vfill}
\begin{block}{A variadic value printer}
\begin{lstlisting}
template <typename T, typename ... Args> // Args is variadic template parameter
void print_values(T x, Args ... args);   // args is variadic function parameter
\end{lstlisting}
\end{block}

\end{itemize}
\end{frame}

\begin{frame}[t,fragile]{Parameter packs and argument packs}
\begin{itemize}
  \item A \textemph{template parameter pack} is a template parameter that
        accepts zero or more template arguments.

  \pause
  \item A \textemph{function argument pack} is a function argument of
        template parameter pack type.

\pause
\begin{block}{Packs in practice}
\begin{lstlisting}[escapechar=@]
template <typename T, typename ... Args>
void print_values(T X, Args ... args);

void f() {
  print_values(42, "Daniel", "Bjarne"s, 9.5);@\pause@
  // T = int. Parameter pack Args = char const *, std::string, double@\pause@
  // x = 42. Argument pack args = "Daniel", "Bjarne"s, 9.5
};
\end{lstlisting}
\end{block}
\end{itemize}
\end{frame}

\begin{frame}[t,fragile]{Operations on packs}
\begin{itemize}
  \item \textbad{Only} two operations defined on packs.

    \mode<presentation>{\vfill\pause}
    \item \textmark{Unpack} or \textmark{expand}: \cppid{\ldots}.
      \begin{itemize}
        \item Can be applied to both parameter and argument packs.
        \item \cppid{Args...} expands a parameter pack.
        \item \cppid{args...} expands an argument pack.
      \end{itemize}

    \mode<presentation>{\vfill\pause}
    \item Count the elements in a pack: \cppkey{sizeof...(x)}.
      \begin{itemize}
        \item Can be applied to both paramter and argument packs.
        \item \cppkey{sizeof...(}\cppid{Args}\cppid{)}.
        \item \cppkey{sizeof...(}\cppid{args}\cppid{)}.
      \end{itemize}
 
  \mode<presentation>{\vfill\pause}
  \item \textbad{Note}: Usually in combination with \textmark{compile-time recursion}.
\end{itemize}
\end{frame}

\begin{frame}[t,fragile]{Printing values}

\begin{columns}[T]

\column{.45\textwidth}
\begin{block}{Printing a list of values}
\begin{lstlisting}
void print_values() { }

template <typename T, typename ... Args>
void print_values(T x, Args ... args) {
  std::cout << x;
  if (sizeof...(args) > 0) {
    std::cout << " , ";
    print_values(args...);
  }
  else {
    std::cout << "\n";
  }
}
\end{lstlisting}
\end{block}

\column{.55\textwidth}
\begin{block}{Invoking value printer}
\begin{lstlisting}
void f() {
  print_values(); // Call non-template

  print_values(42);
  // print_values<int>(42);

  print_values(42, "Daniel");
  // print_values<int,char const*>(42, "Daniel");

  print_values(42, 3.5, 2.4f);
  // print_values<int,double,float>(42, 3.5, 2.4f);
}
\end{lstlisting}
\end{block}

\end{columns}
\end{frame}

\begin{frame}[t,fragile]{Call chain}
\begin{lstlisting}[escapechar=@]
  print_values(42, 3.5, 2.4f);
      std::cout << 42;
      std::cout << " , ";
      print_values(3.5, 2.4f);@\pause@
          std::cout << 3.5;
          std::cout << " , ";
          print_values(2.4f);@\pause@
              std::cout << 2.4f;
              std::cout << '\n';
\end{lstlisting}

\mode<presentation>{\vfill\pause}
\begin{itemize}
  \item \textbad{Observation}: The non-template version is never invoked,
        but must be defined.
    \begin{itemize}
      \item Both branches of the \cppkey{if} \textmark{must compile}.
    \end{itemize}
\end{itemize}
\end{frame}
