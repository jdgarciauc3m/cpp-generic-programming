\subsection{A safe print function}

\begin{frame}[t,fragile]

\begin{columns}[T]

\column{.6\textwidth}
\begin{block}{Implementing a type-safe print}
\begin{lstlisting}
template <typename T, typename... Args>
void my_print(std::string_view str, T first, Args... args) {
  auto pos = str.find("{}");
  std::cout << str.substr(0, pos);
  if (pos == std::string_view::npos) { 
    throw std::invalid_argument("Extra argument"); 
  }
  std::cout << first;
  if constexpr (sizeof...(Args) > 0) {
    my_print(str.substr(pos + 2), args...);
  } else {
    if ( str.find("{}", pos+2) != std::string_view::npos) {
      throw std::invalid_argument("Missing argument");
    }
    std::cout << str.substr(pos + 2);
  }
}
\end{lstlisting}
\end{block}

\pause
\column{.4\textwidth}
\begin{block}{Using the safe print}
\begin{lstlisting}
void f() try {
  // OK
  my_print("Name {}. Birth year = {}. Height ={}.\n", 
      "Daniel", 1969, 1.82);

  // Exception: Extra argument
  my_print("Name {}. Birth year = {}. Height ={}.\n", 
      "Daniel", 1969, 1.82, 4);

  // Exception: Missing argument
  my_print("Name {}. Birth year = {}. Height ={}.\n", 
      "Daniel", 1969);
} catch (std::exception & exc) { 
  std::cerr << exc.what() << '\n'; 
}
\end{lstlisting}
\end{block}

\end{columns}
\end{frame}
