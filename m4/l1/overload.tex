\subsection{Overload resolution}

\begin{frame}[t,fragile]{Variadic and non-variadic templates}
  \begin{itemize}
    \item If two function templates only differ in its trailing
          parameter pack, the non variadic template is preferred.
  \end{itemize}

\begin{columns}[T]

\column{.45\textwidth}
\begin{block}{Printing a list of values}
\begin{lstlisting}
template <typename T>
void print_values(T x) {
  std::cout << x << '\n';
}

template <typename T, typename ... Args>
void print_values(T x, Args ... args) {
  std::cout << x << " , ";
  print_values(args...);
}
\end{lstlisting}
\end{block}

\column{.55\textwidth}
\begin{block}{Invoking a value printer}
\begin{lstlisting}
void f() {
  print_values(); // Error

  print_values(42);
  // print_values<int>(42);

  print_values(42, "Daniel");
  // print_values<int,char const*>(42, "Daniel");

  print_values(42, 3.5, 2.4f);
  // print_values<int,double,float>(42, 3.5, 2.4f);
}
\end{lstlisting}
\end{block}

\end{columns}
\end{frame}

\begin{frame}[t,fragile]{Call chain}
\begin{itemize}
  \item \cppid{print\_values(42, 3.5, 2.4f)}
    \begin{itemize}
      \item \pause \cppid{std::cout <{}< 42 <{}< }\cppstr{" , "}
      \item \pause \cppid{print\_values(3.5, 2.4f)}
      \item \pause \cppid{std::cout <{}< 3.5 <{}< }\cppstr{" , "}
      \item \pause \cppid{print\_values(2.4f)}
      \item \pause \cppid{std::cout <{}< 2.4f <{}< }\cppstr{'\textbackslash{}n'}
    \end{itemize}

  \mode<presentation>{\vfill\pause}
  \item \textgood{Advantage}: Simpler code.

  \mode<presentation>{\vfill}
  \item \textbad{Drawback}: Does not support 0-arguments version.
    \begin{itemize}
      \item It can be solved with a third overload.
    \end{itemize}
\end{itemize}
\end{frame}

\begin{frame}[t,fragile]{Unifying the variadic and non-variadic cases}

\begin{columns}[T]

\pause
\column{.5\textwidth}
\begin{block}{First try}
\begin{lstlisting}
template <typename T, typename... Args>
void print_values(T x, Args... args) {
  std::cout << x << " , ";
  if  (sizeof...(Args) > 0) { 
    print_values(args...); // Error
  }
}
\end{lstlisting}
\end{block}

\pause
\column{.5\textwidth}
\begin{block}{Using compile-time if}
\begin{lstlisting}[escapechar=@]
template <typename T, typename... Args>
void print_values(T x, Args... args) {
  std::cout << x << " , ";
  if @\color{red}constexpr@ (sizeof...(Args) > 0) { 
    print_values(args...); 
  }
}
\end{lstlisting}
\end{block}
\end{columns}
\end{frame}
