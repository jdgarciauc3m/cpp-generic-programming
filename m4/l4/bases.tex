\subsection{Variadic using declarations}

\begin{frame}[t,fragile]{Using declarations}
\begin{itemize}
  \item A \textemph{using declaration} in a derived class 
        injects an overload set from the base class.
\end{itemize}

\begin{columns}[T]
\column{.2\textwidth}
\column{.2\textwidth}
\begin{lstlisting}
class A {
public:
  void f();
};
\end{lstlisting}

\column{.25\textwidth}
\begin{lstlisting}
class B {
public:
  void g();
};
\end{lstlisting}

\column{.35\textwidth}
\begin{lstlisting}
class D : public A, public B {
public:
  using A::f;
  using B::g;
};
\end{lstlisting}
\column{.2\textwidth}

\end{columns}

\mode<presentation>{\vfill\pause}
\begin{itemize}
  \item A using declaration may take a comma separated list.
\begin{lstlisting}
class D : public A, public B {
public:
  using A::f, B::g;
};
\end{lstlisting}
\end{itemize}
\end{frame}

\begin{frame}[t,fragile]{Using and variadic bases}
\begin{itemize}
  \item The comma separated list of a \textmark{using declaration}
        may be variadic pack.
\begin{lstlisting}
template <typename ... T>
class A : public T... {
public:
  using T::print ...; // Inherit all overloads
};
\end{lstlisting}

  \mode<presentation>{\vfill\pause}
  \item Having two bases with the same overload would be ambiguous.

\begin{columns}[T]

\column{.1\textwidth}
\column{.25\textwidth}
\begin{lstlisting}
class B1 {
public:
  void print();
  void print(int x);
};
\end{lstlisting}
\column{.3\textwidth}
\begin{lstlisting}
class B2 {
public:
  void print();
  void print(double x);
};
\end{lstlisting}
\column{.45\textwidth}
\begin{lstlisting}
void f() {
  A<B1,B2> a;
  a.print(5); // B1::print(int)
  a.print(5.0); // B2.print(double)
  a.print(); // Error: Ambiguous call
}
\end{lstlisting}
\column{.1\textwidth}
\end{columns}
\end{itemize}
\end{frame}

\begin{frame}[t,fragile]{Overload sets and lambdas}
\begin{itemize}
  \item We can generate a function object with overloaded \cppkey{operator()}
        from a set of lambdas by inheriting from them.
\end{itemize}

\mode<presentation>{\vfill\pause}
\begin{columns}[T]

\column{.5\textwidth}
\begin{block}{An overload set}
\begin{lstlisting}
template <typename... T>
struct overload : T... {
  using T::operator()...;
};

template <typename ... T>
overload(T...) -> overload<T...>;
\end{lstlisting}
\end{block}

\pause
\column{.5\textwidth}
\begin{block}{Usage}
\begin{lstlisting}
void f() {
  overload fun{
    [](int x) { std::cout << "int:" << x; },
    [](double x) { std::cout << "double:" << x; },
    [](std::string s) { std::cout << "string:" << s; }
  };
  fun(2);
  fun("hello");
  fun(3.5);
}

\end{lstlisting}
\end{block}

\end{columns}
\end{frame}
