\subsection{Indexing}

\begin{frame}[t,fragile]{Integer sequences}
\begin{itemize}
  \item The standard library provides types for index sequences.
\begin{lstlisting}
template <typename T, T ... Is>
class integer_sequence;
\end{lstlisting}
    \begin{itemize}
      \item \textmark{Members}:
        \begin{itemize}
          \item \cppid{value\_type}: The type \cppid{T}.
          \item \cppid{size()}: Returns the number of values.
        \end{itemize}
    \end{itemize}

\begin{columns}[T]

\column{.5\textwidth}
\begin{block}{Sequence printer}
\begin{lstlisting}
template <typename T, T ... Is>
void print_seq(std::integer_sequence<T,Is...>) {
  std::cout << "[";
  ((std::cout << Is << ' '),...);
  std::cout << "]\n";
}
\end{lstlisting}
\end{block}

\column{.5\textwidth}
\begin{block}{Printing a sequence}
\begin{lstlisting}
void f() {
  std::integer_sequence<int, 1, 3, 5, 7> s1;
  print_seq(s1);
  print_seq(std::integer_sequence<long,2,4>{});
}
\end{lstlisting}
\end{block}
\end{columns}

\end{itemize}
\end{frame}

\begin{frame}[t,fragile]{Helpers for integer sequences}
\begin{itemize}
  \item \cppid{std::index\_sequence<std::size\_t ... Is>}: 
        Integer sequence for values of type \cppid{std::size\_t}.

  \mode<presentation>{\vfill\pause}
  \item \cppid{std::make\_integer\_sequence<typename T, T N>}:
        An integer sequence of type \cppid{T} for values \cppid{0},...,\cppid{N-1}.

  \mode<presentation>{\vfill\pause}
  \item \cppid{std::make\_index\_sequence<std::size\_t N>}:
        An integer sequence of type \cppid{std::size\_t} for values \cppid{0},...,\cppid{N-1}.

  \mode<presentation>{\vfill\pause}
  \item \cppid{std::index\_sequence\_for<typename ... T>}:
        Equivalente a \cppid{std::make\_index\_sequence<}\cppkey{sizeof}\cppid{...(T)>}
  
\end{itemize}
\end{frame}

\begin{frame}[t,fragile]{Using index sequences for printing tuples}
\begin{block}{Printing a tuple}
\begin{lstlisting}
template <typename TPL, std::size_t ... Is>
void print_helper(TPL const & tpl, std::index_sequence<Is...>) {
  ((std::cout << (Is==0 ? "" : ", ") << std::get<Is>(tpl)), ...);
}

template <typename ... T>
void print(std::tuple<T...> const & tpl) {
  std::cout << "[";
  print_helper(tpl, std::index_sequence_for<T...>{});
  std::cout << "]\n";
}
\end{lstlisting}
\end{block}
\end{frame}
