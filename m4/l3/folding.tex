\subsection{Expression folding}

\begin{frame}[t,fragile]{Variadic addition}
\begin{itemize}
  \item A \textmark{variadic template function} for \cppid{add}:

\begin{block}{A variadic addition}
\begin{lstlisting}
template <typename T1>
auto add(T1 v1) { return v1; }

template <typename T1, typename ... Ts>
auto add(T1 x1, Ts ... xs) {
  return x1 + add(xs...);
}

void f() {
  auto x = add(1, 2.0);
  auto y = add(1L, 1.5f, 2LL);
  //...
}
\end{lstlisting}
\end{block}
\end{itemize}
\end{frame}

\begin{frame}[t,fragile]{Expression folding}
\begin{itemize}
  \item With \textmark{expression folding} a binary operator
        can be applied to a parameter pack.

\begin{columns}[T]

\column{.5\textwidth}
\begin{block}{A folded addition}
\begin{lstlisting}
template <typename ... T>
auto add(T ... x) {
  return (x + ...);
}
\end{lstlisting}
\end{block}

\column{.5\textwidth}
\begin{block}{Invoking folded addition}
\begin{lstlisting}
void f() {
  auto x1 = add(1L, 1.5F, 2LL); // OK
  auto x2 = add(3); // OK
}
\end{lstlisting}
\end{block}

\end{columns}
\end{itemize}
\end{frame}

\begin{frame}[t,fragile]{Folding default value}
\mode<presentation>{\vspace{-.75em}}
\begin{itemize}
  \item A folding expression with an empty pack is \textbad{ill-formed}.
    \begin{itemize}
      \item Solvable with an initial value.
    \end{itemize}

\mode<presentation>{\vspace{-.75em}}
\begin{columns}[T]

\column{.5\textwidth}
\begin{block}{A folded addition}
\begin{lstlisting}
template <typename ... T>
auto add(T ... x) { return (x + ... + 0); }
\end{lstlisting}
\end{block}

\column{.5\textwidth}
\begin{block}{Invoking folded addition}
\begin{lstlisting}
void f() { auto x1 = add(); } // OK
\end{lstlisting}
\end{block}

\end{columns}

  \mode<presentation>{\vfill\pause}
  \item Two operators have a default initial value: 
        \cppkey{and} (\cppkey{true}) and \cppkey{or} (\cppkey{false}).
\begin{columns}[T]

\column{.5\textwidth}
\begin{block}{A folded logical and}
\begin{lstlisting}
template <typename ... T>
auto combine(T ... x) { return (x and ... ); }
\end{lstlisting}
\end{block}

\column{.5\textwidth}
\begin{block}{Invoking folded addition}
\begin{lstlisting}
void f() { auto x1 = combine(); } // OK
\end{lstlisting}
\end{block}

\end{columns}


\end{itemize}
\end{frame}

\begin{frame}[t,fragile]{Left and right folding}
\begin{itemize}
  \item \textgood{Left fold}:
    \begin{itemize}
      \item \cppid{(\ldots $\otimes$ args)} \quad $\rightarrow$ \quad 
            $(\ldots((a_1 \otimes a_2) \otimes a_3) \otimes \ldots) \otimes a_n$
      \item \cppid{(x $\otimes$ \ldots $\otimes$ args)} \quad $\rightarrow$ \quad
            $(\ldots((x \otimes a_1) \otimes a_2) \otimes \ldots) \otimes a_n$
    \end{itemize}
  \item \textgood{Right fold}:
    \begin{itemize}
      \item \cppid{(args $\otimes$ \ldots)} \quad $\rightarrow$ \quad
            $a_1 \otimes (\ldots \otimes (a_{n-1} \otimes a_n))$ 
      \item \cppid{(args $\otimes$ \ldots $\otimes$ x)} \quad $\rightarrow$ \quad
            $a_1 \otimes (\ldots \otimes (a_{n-1} \otimes (a_n \otimes x)))$ 
    \end{itemize}

  \mode<presentation>{\vfill\pause}
  \item \textbad{Note}: Some operations might make sense only for one folding.

\begin{columns}[T]

\column{.5\textwidth}
\begin{block}{Using left folding for printing}
\begin{lstlisting}
template <typename ... T>
void print(T ... x) {
  (std::cout << ... << x) << '\n';
}
\end{lstlisting}
\end{block}

\pause
\column{.5\textwidth}
\begin{block}{Cannot use right folding for printing}
\begin{lstlisting}
template <typename ... T>
void print(T ... x) {
  (std::cout << x << ...) << '\n'; // Error
}
\end{lstlisting}
\end{block}

\end{columns}
\end{itemize}
\end{frame}

\begin{frame}[t,fragile]{Improving a variadic print}
\mode<presentation>{\vspace{-.75em}}
\begin{itemize}
  \item \textbad{Problem}: We cannot add extra operands.
  \item \textgood{Solution}: An intermediate type.
\end{itemize}

\mode<presentation>{\vspace{-.75em}}
\pause
\begin{columns}[T]

\column{.57\textwidth}
\begin{block}{A formatting type}
\begin{lstlisting}
template <typename T>
class with_extra {
public:
  with_extra(T const & value) : value_{value} {}

  friend std::ostream & 
  operator<<(std::ostream & os, with_extra const & other) {
    return os << other.value_ << ' ';
  }
private:
  T const & value_;
};
\end{lstlisting}
\end{block}

\pause
\column{.45\textwidth}
\begin{block}{Using the formatter}
\begin{lstlisting}
template <typename ... T>
void print_with_extra(T ... x) {
  (std::cout << ... << with_extra{x}) << '\n';
}
\end{lstlisting}
\end{block}

\end{columns}

\end{frame}
