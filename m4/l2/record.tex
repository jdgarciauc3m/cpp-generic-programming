\subsection{A generic record}

\begin{frame}[t,fragile]{A generic record type}
\begin{itemize}
  \item A variadic class template.
    \begin{itemize}
      \item One data member for each template parameter.
    \end{itemize}

\begin{columns}[T]

\column{.5\textwidth}
\begin{block}{Variadic based code}
\begin{lstlisting}
record<int, std::string, double> rec;
\end{lstlisting}
\end{block}

\column{.5\textwidth}
\begin{block}{Equivalent code}
\begin{lstlisting}
class concrete_record {
  //...
private:
  int member0;
  std::string member1;
  double member2;
};
concrete_record rec;
\end{lstlisting}
\end{block}

\end{columns}

\end{itemize}
\end{frame}

\begin{frame}[t,fragile]
\begin{block}{A minimal variadic record}
\begin{lstlisting}
template<typename ... T>
class record;

template <>
class record<> {};

template <typename T, typename ... U>
class record<T,U...> {
public:
  record(T const & first, U const & ... rest) :
      first_{first}, rest_{rest...} {}

  [[nodiscard]] T const & first() const { return first_; }
  [[nodiscard]] record<U...> const & rest() const { return rest_; }

private:
  T first_;
  record<U...> rest_;
};
\end{lstlisting}
\end{block}
\end{frame}

\begin{frame}[t,fragile]{Adding default construction}
\begin{block}{A record with default construction}
\begin{lstlisting}[escapechar=@]
template <typename T, typename ... U>
class record<T,U...> {
public:
  @\color{red}record() = default;@
  record(T const & first, U const & ... rest) :
      first_{first}, rest_{rest...} {}

  [[nodiscard]] T const & first() const { return first_; }
  [[nodiscard]] record<U...> const & rest() const { return rest_; }

private:
  T first_@\color{red}\{\}@;
  record<U...> rest_@\color{red}\{\}@;
};
\end{lstlisting}
\end{block}
\end{frame}

\begin{frame}[t,fragile]{Printing record contents}

\begin{columns}[T]

\column{.5\textwidth}
\begin{block}{Adding a print function}
\begin{lstlisting}[escapechar=@]
template <typename T, typename ... U>
class record<T,U...> {
public:
  record() = default;
  record(T const & first, U const & ... rest) :
      first_{first}, rest_{rest...} {}

  @\color{red}void print() const;@

private:
  T first_{};
  record<U...> rest_{};
};
\end{lstlisting}
\end{block}

\pause
\column{.5\textwidth}
\begin{block}{Adding a print function}
\begin{lstlisting}
template <typename T, typename ... U>
void record<T,U...>::print() const {
  std::cout << first_;
  if constexpr(sizeof...(U) > 0) {
    std::cout << " , ";
    rest_.print();
  } else {
    std::cout << '\n';
  }
}
\end{lstlisting}
\end{block}

\end{columns}
\end{frame}

\begin{frame}[t,fragile]{Adding indexed access support}
\mode<presentation>{\vspace{-.75em}}
\begin{itemize}
  \item Adding a free function to get a value in the record.

\begin{block}{Using a indexed value accessor}
\begin{lstlisting}
void f() {
  record<int, std::string, double> value{1969, "Daniel", 1.82};
  std::cout << std::format("{}: {} ({})\n", get<0>(value), get<1>(value), get<2>(value));
}
\end{lstlisting}
\end{block}

  \mode<presentation>{\vfill\pause}
  \item \textbad{Problem}: A function cannot be partially specialized.


\mode<presentation>{\vspace{-.75em}}
\begin{columns}[T]

\column{.5\textwidth}
\begin{block}{Defining the accessor}
\begin{lstlisting}
template <int I, typename ... T>
auto get(record<T...> const & rec) {
  return get<I-1>(rec.rest());
}
\end{lstlisting}
\end{block}

\pause
\column{.5\textwidth}
\begin{block}{Trying to partial specialize}
\begin{lstlisting}[escapechar=@]
template <typename T, typename ... U>
auto @\color{red}get<0>@(record<T,U...> const & rec) { // Error
  return rec.first();
}
\end{lstlisting}
\end{block}

\end{columns}
\end{itemize}
\end{frame}

\begin{frame}[t,fragile]{Achieving function partial specialization}
\mode<presentation>{\vspace{-.75em}}
\begin{itemize}
  \item Use a helper type that can be partially specialized.

\mode<presentation>{\vspace{-.75em}}
\begin{columns}[T]

\column{.5\textwidth}
\begin{block}{Defining helper type}
\begin{lstlisting}[escapechar=@]
template <int I>
struct get_helper {
  template <typename ... T>
  static auto get(record<T...> const & rec) {
    return get_helper<I-1>::get(rec.rest());
  }
};
@\pause@
template <>
struct get_helper<0> {
  template <typename ... T>
  static auto get(record<T...> const & rec) {
    return rec.first();
  }
};
\end{lstlisting}
\end{block}

\pause
\column{.4\textwidth}
\begin{block}{Using the helper type}
\begin{lstlisting}
template <int I, typename ... T>
auto get(record<T...> const & rec) {
  return get_helper<I>::get(rec);
}
\end{lstlisting}
\end{block}

\end{columns}

\end{itemize}
\end{frame}


\begin{frame}[t,fragile]{Simplifying accessor}
\begin{columns}[T]

\column{.5\textwidth}
\begin{block}{Using if}
\begin{lstlisting}[escapechar=@]
template<int I, typename ... T>
auto get(record<T...> const &rec) {
  if  (I > 0) {
    return get<I - 1>(@\color{red}rec.rest()@); // Error
  } else {
    return rec.first();
  }
}
\end{lstlisting}
\end{block}

\pause
\column{.5\textwidth}
\begin{block}{Using compile-time if}
\begin{lstlisting}[escapechar=@]
template<int I, typename ... T>
auto get(record<T...> const &rec) {
  if @\color{red}constexpr@ (I > 0) {
    return get<I - 1>(rec.rest());
  } else {
    return rec.first();
  }
}
\end{lstlisting}
\end{block}
\end{columns}
\end{frame}
