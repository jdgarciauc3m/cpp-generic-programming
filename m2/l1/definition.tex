\subsection{Generic class definition}

\begin{frame}[t,fragile]{Generic classes}
\begin{itemize}
  \item A \textemph{generic class} is a class that has one or more
        \textmark{generic parameters}.

  \mode<presentation>{\vfill\pause}
  \item Every \textmark{type parameter} of a generic class may be:

\begin{columns}[T]

\pause
\column{.5\textwidth}
    \begin{itemize}
      \item An \textmark{unconstrained} type marked with keywords \cppkey{typename} or \cppkey{class}.
    \end{itemize}

\begin{block}{Unconstrained generic class}
\begin{lstlisting}
template <typename T>
class numeric_vector {
 // Members...
};
\end{lstlisting}
\end{block}

\pause
\column{.5\textwidth}
    \begin{itemize}
      \item A \textmark{constrained} type marked with a \textemph{concept name}.
    \end{itemize}
\begin{block}{Constrained generic class}
\begin{lstlisting}
template <std::floating_point T>
class numeric_vector {
 // Members...
};
\end{lstlisting}
\end{block}

\end{columns}

\end{itemize}
\end{frame}


\begin{frame}[t,fragile]{Some useful concepts}
\begin{itemize}
  \item When defining class templates there are some useful concepts to be used:

  \begin{itemize}

    \mode<presentation>{\vfill}
    \item \cppid{std::copyable}: Types that are movable and can be copied.

    \mode<presentation>{\vfill}
    \item \cppid{std::default\_initializable}: Types that can be default constructed.

    \mode<presentation>{\vfill}
    \item \cppid{std::semiregular}: Types that are both copyable and default constructible.

    \mode<presentation>{\vfill}
    \item \cppid{std::equality\_comparable}: Types that support equality comparison.
    \begin{itemize}
      \item If two objects are equal the equality comparison operator returns \cppkey{true}.
      \item Equality comparison is reflexive, symmetric and transitive.
    \end{itemize}

    \mode<presentation>{\vfill}
    \item \cppid{std::regular}: Types that are both \cppid{std::semiregular} and \cppid{equality\_comparable}.

  \end{itemize}

\end{itemize}
\end{frame}


\begin{frame}[t,fragile]{Defining a generic class}

\mode<presentation>{\vspace{-1em}}
\begin{columns}[T]

\column{.6\textwidth}
\begin{block}{A generic fixed vector}
\begin{lstlisting}[escapechar=@]
template <std::semiregular @\color{red}T@>
class fixed_vector {
public:
  fixed_vector() = default;
  fixed_vector(int sz);

  ~fixed_vector() noexcept = default;

  fixed_vector(fixed_vector const & other);
  fixed_vector & operator=(fixed_vector const & other);
  fixed_vector(fixed_vector && other) noexcept = default;
  fixed_vector & 
  operator=(fixed_vector && other) noexcept = default;
  //...
\end{lstlisting}
\end{block}

\column{.4\textwidth}
\begin{block}{A generic fixed vector}
\begin{lstlisting}[escapechar=@]
  int size() const { return size_; }
  int capacity() const { return capacity_; }

  [[nodiscard]] @\color{red}T@ & operator[](int i);
  [[nodiscard]] @\color{red}T@ operator[](int i) const;

  void push_back(@\color{red}T@ const & x);

private:
  int capacity_ = 0;
  int size_ = 0;
  std::unique_ptr<@\color{red}T[]@> buffer_{};
};
\end{lstlisting}
\end{block}

\end{columns}
\end{frame}

\begin{frame}[t,fragile]{Naming the class template}
\begin{itemize}
  \item Naming the class template inside the definition does not need to include its parameters.

\begin{columns}[T]

\column{.5\textwidth}
\begin{block}{Naming the class template}
\begin{lstlisting}[escapechar=@]
template <std::semiregular @\color{red}T@>
class fixed_vector {
public:
  //...
  fixed_vector(@\color{red}fixed\_vector@ const & other)
  //...
};
\end{lstlisting}
\end{block}

\column{.5\textwidth}
\begin{block}{Equivalent}
\begin{lstlisting}[escapechar=@]
template <std::semiregular T>
class fixed_vector {
public:
  //...
  fixed_vector(@\color{red}fixed\_vector<T>@ const & other)
  //...
};
\end{lstlisting}
\end{block}

\end{columns}
\end{itemize}
\end{frame}


\begin{frame}[t,fragile]{Defining memeber functions in class}
\begin{block}{Constructors}
\begin{lstlisting}[escapechar=@]
template <std::semiregular @\color{red}T@>
class fixed_vector {
public:
  //...
  fixed_vector(int n) :
      capacity_{n},
      buffer_{std::make_unique<@\color{red}T[]@>(gsl::narrow<std::size_t>(n))} {
    Expects(n > 0);
    Ensures(capacity_ == n);
    Ensures(size_ == 0);
    Ensures(buffer_ != nullptr);
  }
  //...
}
\end{lstlisting}
\end{block}
\end{frame}

\begin{frame}[t,fragile]{Naming the class template}
\mode<presentation>{\vspace{-1em}}
\begin{itemize}
  \item However, naming the class outside the class definition needs to include
        its parameters.

\begin{block}{Naming the class template}
\begin{lstlisting}[escapechar=@]
template<std::semiregular @\color{red}T@>
@\color{red}fixed\_vector<T>@ & @\color{red}fixed\_vector<T>@::operator=(const @\color{red}fixed\_vector@ & other) {
  if (this == &other) { return *this; }
  auto aux = std::make_unique<@\color{red}T[]@>(other.capacity_);
  std::copy(other.get().other.size(), aux.get());
  capacity_ = other.capacity_;
  size_ = other.size_;
  buffer_ = std::move(other.buffer_);

  Ensures(capacity_ == other.capacity_);
  Ensures(size_ == other.size_);
  Ensures(buffer_ != nullptr);
}
\end{lstlisting}
\end{block}

\end{itemize}
\end{frame}
