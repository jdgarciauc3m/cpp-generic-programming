\subsection{Partial specialization and multiple parameters}

\begin{frame}[t,fragile]{Generalizing a strategy}
\begin{itemize}
  \item A \cppid{fixed\_vector} may be implemented with different 
        storage strategies:
    \begin{itemize}
      \item \cppid{allocated\_storage}: The storage buffer is dynamically allocated
            on the free store.
        \begin{itemize}
          \item Size determined at run time.
          \item Small overhead in allocation and access.
        \end{itemize}

      \item \cppid{local\_storage}: The storage is a local buffer of a fixed size.
        \begin{itemize}
          \item Size predefined at compile time.
          \item No overhead in allocation or access.
        \end{itemize}
    \end{itemize}
\end{itemize}

\begin{columns}

\column{.5\textwidth}
\begin{block}{Strategy based \textbf{fixed\_vector}}
\begin{lstlisting}
template <std::semiregular T, typename S>
class fixed_vector {
  //...
};
\end{lstlisting}
\end{block}

\column{.5\textwidth}
\begin{itemize}
  \item \cppid{T}: Element type.
  \item \cppid{S}: Storage type.
\end{itemize}

\end{columns}

\end{frame}

\begin{frame}[t,fragile,shrink=20]{Implementing the storage strategies}
\begin{block}{A dynamic memory storage}
\begin{lstlisting}
template<std::semiregular T>
class allocated_storage {
public:
  explicit allocated_storage(int n) : capacity_{n}, buffer_{std::make_unique<T[]>(gsl::narrow<std::size_t>(n))} {
    Expects(n > 0);
  }

  [[nodiscard]] int capacity() const { return capacity_; }

  T operator[](int i) const {
    Expects(i >= 0 and i < capacity_);
    return buffer_[gsl::narrow<std::size_t>(i)];
  }

  T &operator[](int i) {
    Expects(i >= 0 and i < capacity_);
    return buffer_[gsl::narrow<std::size_t>(i)];
  }

private:
  int capacity_;
  std::unique_ptr<T[]> buffer_;
};
\end{lstlisting}
\end{block}
\end{frame}

\begin{frame}[t,fragile,shrink=20]{Implementing the storage strategies}
\begin{block}{A local memory storage}
\begin{lstlisting}
template<std::semiregular T>
class local_storage {
public:
  local_storage() = default;

  [[nodiscard]] int capacity() const { return max_size; }

  T operator[](int i) const {
    Expects(i >= 0 and i < max_size);
    return buffer_[gsl::narrow<std::size_t>(i)];
  }

  T &operator[](int i) {
    Expects(i >= 0 and i < max_size);
    return buffer_[gsl::narrow<std::size_t>(i)];
  }

private:
  static constexpr int max_size = 32;
  std::array<T, max_size> buffer_;
};
\end{lstlisting}
\end{block}
\end{frame}

\begin{frame}[t,fragile]{Differences in strategies}
\begin{itemize}
  \item Both \textmark{storage strategies} are quite similar:
    \begin{itemize}
      \item They have some storage where data can be stored.
      \item They offer a \cppid{capacity()} member.
      \item They offer \cppkey{operator[]} to access elements.
    \end{itemize}

  \mode<presentation>{\vfill\pause}
  \item However, there are small differences:
    \begin{itemize}
      \item Constructor for \cppid{allocated\_storage} takes a size.
      \item Constructor for \cppid{local\_storage} takes no parameter.
    \end{itemize}

  \mode<presentation>{\vfill\pause}
  \item \textemph{Idea}:
    \begin{itemize}
      \item Write a \cppid{fixed\_vector} for general case (\cppid{allocated\_storage}).
      \item Partially specialize for \cppid{local\_storage}.
    \end{itemize}
\end{itemize}
\end{frame}

\begin{frame}[t,fragile,shrink=10]{Generalizing for strategies}
\begin{block}{A generalized \textbf{fixed\_vector}}
\begin{lstlisting}
template<std::semiregular T, typename S>
class fixed_vector {
public:
  explicit fixed_vector(int n) : storage_{n} {}

  T operator[](int i) const { return storage_[i]; }
  T &operator[](int i) { return storage_[i]; }

  [[nodiscard]] int capacity() const { return storage_.capacity(); }
  [[nodiscard]] int size() const { return size_; }

  void push_back(T x);

  template <std::semiregular T2, typename S2>
  friend std::ostream &operator<<(std::ostream &os, fixed_vector<T2,S2> const &v);

private:
  int size_ = 0;
  S storage_;
};
\end{lstlisting}
\end{block}
\end{frame}

\begin{frame}[t,fragile]{Generalizing for strategies}
\begin{block}{Implementing members}
\begin{lstlisting}
template <std::semiregular T, typename S>
void fixed_vector<T,S>::push_back(T x) {
  Expects(size() < capacity());
  storage_[size_++] = x;
}

template <std::semiregular T, typename S>
std::ostream &operator<<(std::ostream &os, fixed_vector<T,S> const &v) {
  if (v.size_ <= 0) { return os; }
  os << v.storage_[0];
  for (int i = 1; i < v.size_; ++i) {
    os << " , " << v.storage_[i];
  }
  return os;
}
\end{lstlisting}
\end{block}
\end{frame}

\begin{frame}[t,fragile,shrink=10]{Specializing for local storage}
\begin{block}{A specialized \textbf{fixed\_vector}}
\begin{lstlisting}
template<std::semiregular T>
class fixed_vector<T, local_storage<T>> {
public:
  explicit fixed_vector(int n) {
    Expects(n>=0 and n<storage_.capacity());
  }

  T operator[](int i) const { return storage_[i]; }
  T &operator[](int i) { return storage_[i]; }

  [[nodiscard]] int capacity() const { return storage_.capacity(); }
  [[nodiscard]] int size() const { return size_; }

  void push_back(T x);

  template <std::semiregular T2, typename S2>
  friend std::ostream &operator<<(std::ostream &os, fixed_vector<T2,S2> const &v);

private:
  int size_ = 0;
  local_storage<T> storage_;
};

\end{lstlisting}
\end{block}
\end{frame}
