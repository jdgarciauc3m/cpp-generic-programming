\subsection{Partial specialization}

\begin{frame}[t,fragile]{Partially specializing a class}
\begin{itemize}
  \item A class can be partially specialized:
    \begin{itemize}
      \item The class is specialized.
      \item But it still depends on some generic parameter.
    \end{itemize}

\begin{block}{A partial specialization for pointers}
\begin{lstlisting}[escapechar=@]
template <std::semiregular T>
class fixed_vector<@\color{red}T*@> {
  //...
};
\end{lstlisting}
\end{block}

\end{itemize}
\end{frame}

\begin{frame}[t,fragile,shrink=20]{Partially specializing for pointers}
\mode<presentation>{\vspace{-1em}}

\begin{columns}[T]

\column{.5\textwidth}
\begin{block}{A \textbf{fixed\_vector} of pointers}
\begin{lstlisting}[escapechar=@]
template<std::semiregular T>
class fixed_vector<@\color{red}T *@> {
public:
  fixed_vector() = default;
  explicit fixed_vector(int n);

  ~fixed_vector() noexcept = default;

  fixed_vector(fixed_vector const &other);
  fixed_vector<T *> &operator=(fixed_vector const &other);
  fixed_vector(fixed_vector &&) noexcept = default;
  fixed_vector &operator=(fixed_vector &&) noexcept = default;

  @\color{red}T *@ &operator[](int i) {
    Expects(i >= 0 and i < size_);
    return buffer_[gsl::narrow<std::size_t>(i)];
  }
\end{lstlisting}
\end{block}

\column{.5\textwidth}
\begin{block}{A \textbf{fixed\_vector} of pointers}
\begin{lstlisting}[escapechar=@]
  @\color{red}T *@ operator[](int i) const {
    Expects(i >= 0 and i < size_);
    return buffer_[gsl::narrow<std::size_t>(i)];
  }

  [[nodiscard]] int size() const { return size_; }
  [[nodiscard]] int capacity() const { return capacity_; }

  void push_back(@\color{red}T *@ x);

  void serialize(std::ostream &os) const;
  friend std::ostream &operator <<<T*>(
      std::ostream &os, 
      fixed_vector const &v);

private:
  int capacity_ = 0;
  int size_ = 0;
  std::unique_ptr<@\color{red}T *@[]> buffer_{};
};
\end{lstlisting}
\end{block}

\end{columns}
\end{frame}

\begin{frame}[t,fragile]{Why partially specializing?}
\begin{itemize}
  \item Partial specialization allows for different behavior.
    \begin{itemize}
      \item Different printing for \cppid{fixed\_vector} of pointers.
    \end{itemize}

  \mode<presentation>{\vfill\pause}
  \item \textbad{Problem}:
    \begin{itemize}
      \item \cppkey{operator<{}<} is a friend function.
      \item Non-member functions are \textmark{free functions}.
      \item A non-member function cannot be partially specialized.
    \end{itemize}

  \mode<presentation>{\vfill\pause}
  \item \textbad{Solution}:
    \begin{itemize}
      \item Write a member function (\cppid{serialize()}) that can be partially specialized.
      \item Invoke \cppid{serialize()} from \cppkey{operator<{}<}.
    \end{itemize}
\end{itemize}
\end{frame}

\begin{frame}[t,fragile]{Serialization for pointers}
\begin{block}{Serialization}
\begin{lstlisting}
template<std::semiregular T>
void fixed_vector<T *>::serialize(std::ostream &os) const {
  auto print_elem = [&os](T *ptr) {
    os << static_cast<void *>(ptr) << " -> ";
    if (ptr) { os << *ptr; }
    else { os << "<null>"; }
  };
  if (size_ <= 0) { return; }
  print_elem(buffer_[0]);
  for (int i = 1; i < size_; ++i) {
    os << " , ";
    print_elem(buffer_[gsl::narrow<std::size_t>(i)]);
  }
}
\end{lstlisting}
\end{block}
\end{frame}
