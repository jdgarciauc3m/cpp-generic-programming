\subsection{Specializing a generic class}

\begin{frame}[t,fragile]{Class template specialization}
\begin{itemize}
  \item A class template can be \textemph{specialized} for a \textmark{concrete type}.
    \begin{itemize}
      \item A different implementation for a given argument.
      \item Usually for optimization.
    \end{itemize}

\begin{block}{Specializing a class template}
\begin{lstlisting}
template <std::semiregular T>
class fixed_vector {
  //...
};

template <>
class fixed_vector<std::string> {
  //...
};
\end{lstlisting}
\end{block}

\end{itemize}
\end{frame}


\begin{frame}[t,fragile]{Specializing for optimization}
\begin{itemize}
  \item Ideas for optimizing \cppid{fixed\_vector<std::string>}:
    \begin{itemize}
      \item \cppkey{const} version of \cppkey{operator[]} might return a \textemph{const reference}.
\begin{lstlisting}
std::string const & operator[](int i);
\end{lstlisting}
      \item \cppid{push\_back()} might take a \textemph{const reference}.
\begin{lstlisting}
void push_back(std::string const & x);
\end{lstlisting}
    \end{itemize}

  \item \textbad{Notes}:
    \begin{itemize}
      \item A class specialization may have \textmark{different data members}.
      \item A class specialization may have \textmark{different member functions}.
      \item Member functions may have \textmark{different signature} 
            in an specialization.
    \end{itemize}
\end{itemize}
\end{frame}

\begin{frame}[t,fragile,shrink=20]{Class specialization}
\mode<presentation>{\vspace{-1em}}
\begin{columns}[T]

\column{.5\textwidth}
\begin{block}{Specializing \textbf{fixed\_vector}}
\begin{lstlisting}[escapechar=@]
template<>
class fixed_vector<@\color{red}std::string@> {
public:
  fixed_vector() = default;
  explicit fixed_vector(int n);

  ~fixed_vector() noexcept = default;

  fixed_vector(fixed_vector<std::string> const &other);
  fixed_vector &operator=(fixed_vector const &other);
  fixed_vector(fixed_vector &&) noexcept = default;
  fixed_vector &operator=(fixed_vector &&) noexcept = default;

  @\color{red}std::string@ & operator[](int i) {
    Expects(i >= 0 and i < size_);
    return buffer_[gsl::narrow<std::size_t>(i)];
  }

\end{lstlisting}
\end{block}

\column{.5\textwidth}
\begin{block}{Specializing \textbf{fixed\_vector}}
\begin{lstlisting}[escapechar=@]
  @\color{red}std::string@ const & operator[](int i) const {
    Expects(i >= 0 and i < size_);
    return buffer_[gsl::narrow<std::size_t>(i)];
  }

  [[nodiscard]] int size() const { return size_; }
  [[nodiscard]] int capacity() const { return capacity_; }

  void push_back(std::string const & x);

  friend std::ostream & operator<<(std::ostream &os, 
                                   fixed_vector const &v);

private:
  int capacity_ = 0;
  int size_ = 0;
  std::unique_ptr<@\color{red}std::string@[]> buffer_{};
};
\end{lstlisting}
\end{block}

\end{columns}
\end{frame}

\begin{frame}[t,fragile]{Specialized member functions}
\begin{itemize}
  \item A member function from a class specialization does not use any
        \cppkey{template<>} prefix.

\mode<presentation>{\vfill}
\begin{block}{Specializing \textbf{std::fixed\_vector::push\_back()}}
\begin{lstlisting}[escapechar=@]
void fixed_vector<@\color{red}std::string@>::push_back(@\color{red}std::string const \&@ x) {
  Expects(size_ >= 0 and size_ < capacity_);
  buffer_[gsl::narrow<std::size_t>(size_++)] = x;
  Ensures(size_ <= capacity_);
}
\end{lstlisting}
\end{block}
\end{itemize}
\end{frame}
