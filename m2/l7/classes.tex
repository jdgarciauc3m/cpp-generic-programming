\begin{frame}[t,fragile]{Generic classes with NTTP}
\begin{itemize}
  \item A generic class may have a non-type template paramter.
    \begin{itemize}
      \item It represents a value that is known at compile time.
    \end{itemize}

\begin{block}{A generic \textbf{local\_storage}}
\begin{lstlisting}[escapechar=@]
template<std::semiregular T, @\color{red}std::size\_t max\_size@>
class local_storage {
  //...
};
\end{lstlisting}
\end{block}

  \item The NTTP may be used:
    \begin{itemize}
      \item In any declarartion inside the class.
      \item In any definition for a class member.
    \end{itemize}
\end{itemize}
\end{frame}


\begin{frame}[t,fragile,shrink=30]{Defining a generic class with NTTP}

\mode<presentation>{\vspace{-1em}}
\begin{columns}[T]

\column{.5\textwidth}
\begin{block}{A local storage with generic size}
\begin{lstlisting}[escapechar=@]
template<std::semiregular T, @\color{red}std::size\_t max\_size@>
class local_storage {
public:
  local_storage() : buffer_{} {}

  [[nodiscard]] int capacity() const { return @\color{red}max\_size@; }

  T operator[](int i) const {
    auto idx = gsl::narrow<std::size_t>(i);
    Expects(idx < @\color{red}max\_size@);
    return buffer_[idx];
  }

  T &operator[](int i) {
    auto idx = gsl::narrow<std::size_t>(i);
    Expects(idx < @\color{red}max\_size@);
    return buffer_[idx];
  }

private:
  T buffer_[@\color{red}max\_size@];
};

\end{lstlisting}
\end{block}

\pause
\column{.5\textwidth}
\begin{block}{Using the local storage}
\begin{lstlisting}[escapechar=@]
void f1() {
  local_storage<double,@\color{red}30@> space;
  space[1] = 3.5;
  std::cout << "space[1] = " << space[1] << '\n';
  double x = space[3];
  std::cout << "space[3] = " << x << '\n';
}
@\pause@
void f2() {
  local_storage<std::string,@\color{red}10@> names;
  local_storage<std::string,@\color{red}4@> seasons;

  if (std::is_same_v<decltype(names), decltype(seasons)>) {
    std::cout << "Same type\n";
  }
  else {
    std::cout << "Different types";
  }
}
\end{lstlisting}
\end{block}

\end{columns}
\end{frame}

\begin{frame}[t,fragile]{Default template parameters}
\begin{itemize}
  \item A template parameter may have a default value also for NNTP.
\end{itemize}

\begin{block}{Storage with default parameters}
\begin{lstlisting}[escapechar=@]
template<std::semiregular T, std::size_t max_size @\color{red}= 32@>
class local_storage {
  //...
};

template<std::semiregular T = double, typename S @\color{red}= local\_storage<T,32>@>
class fixed_vector {
  //...
};
\end{lstlisting}
\end{block}
\end{frame}

\begin{frame}[t,fragile]{Default template parameters and composition}
\mode<presentation>{\vspace{-1em}}
\begin{block}{A generic defaulted fixed vector}
\begin{lstlisting}[basicstyle=\mode<presentation>{\tiny},escapechar=@]
template<@\color{red}std::semiregular T = double, typename S = local\_storage<T,32>@>
class fixed_vector {
public:
  explicit fixed_vector(int n) : storage_{n} {} // Will not work with local_storage

  T operator[](int i) const { return storage_[i]; }
  T &operator[](int i) { return storage_[i]; }

  [[nodiscard]] int capacity() const { return storage_.capacity(); }
  [[nodiscard]] int size() const { return size_; }

  void push_back(T x) {
    Expects(size() < capacity());
    storage_[size_++] = x;
  }

  friend std::ostream &operator<<(std::ostream &os, fixed_vector const &v);

private:
  int size_ = 0;
  @\color{red}S storage\_@;
};
\end{lstlisting}
\end{block}
\end{frame}

\begin{frame}[t,fragile]
\mode<presentation>{\vspace{-1em}}
\begin{block}{A specialized generic defaulted fixed vector}
\begin{lstlisting}[basicstyle=\mode<presentation>{\tiny}, escapechar=@]
template<std::semiregular T, std::size_t max_size> // No defaults
class fixed_vector<T, local_storage<T,max_size>> {
public:
  explicit fixed_vector(int n) {
    Expects(n>=0 and n<storage_.capacity());
  }

  T operator[](int i) const { return storage_[i]; }
  T &operator[](int i) { return storage_[i]; }

  [[nodiscard]] int capacity() const { return storage_.capacity(); }
  [[nodiscard]] int size() const { return size_; }

  void push_back(T x) {
    Expects(size() < capacity());
    storage_[size_++] = x;
  }

  friend std::ostream &operator<<(std::ostream &os, fixed_vector const &v);

private:
  int size_ = 0;
  @\color{red}local\_storage<T,max\_size> storage\_@;
};
\end{lstlisting}
\end{block}
\end{frame}
