\subsection{Using CTAD}

\begin{frame}[t,fragile]{Deducing arguments for function templates}
\begin{itemize}
  \item In C++ template arguments are \textmark{implicitly}
        deduced for function calls.
\begin{block}{Function template implicit deduction}
\begin{lstlisting}[escapechar=@]
template <typename T>
void print_size(T x) { std::cout << "size = " << sizeof(x) << "\n"; }

void g() {
  print_size(42); // print_size<int>(2);
  print_size(1.2); // print_size<double>(1.2)
@\pause@   
  std::complex<double> z{1.0, 2.5};
  print_size(z); // print_size<complex<double>>(z)
@\pause@   
  std::vector<std::map<std::string, int>> v;
  print_size(v); // f<std::vector<std::map<std::string, int>>>(v)
}
\end{lstlisting}
\end{block}
\end{itemize}
\end{frame}

\begin{frame}[t,fragile]{Argument deduction}
\begin{itemize}
  \item Class template arguments can be deduced from constructor arguments.

\begin{block}{Using argument deduction}
\begin{lstlisting}
std::pair p{1969, "Daniel"}; // std::pair<int, char const *>

using namespace std::literals;
std::pair q{1950, "Bjarne"s}; // std::pair<int,std::string>

std::map<int,double> create_dic();
std::tuple t{1, 2.5, create_dic()};// std::tuple<int, double, std::map<int,double>>
\end{lstlisting}
\end{block}

\end{itemize}
\end{frame}

\begin{frame}[t,fragile]{Code simplification}

\begin{block}{Without deduction}
\begin{lstlisting}
std::mutex m1;
std::recursive_mutex m2;
std::shared_lock<std::recursive_mutex> sl{m2};
std::lock_guard<std::mutex, std::shared_lock<std::recursive_mutex>> l{m1,s1};
\end{lstlisting}
\end{block}

  \mode<presentation>{\vfill\pause}
\begin{block}{With deduction}
\begin{lstlisting}
std::mutex m1;
std::recursive_mutex m2;
std::shared_lock sl{m2};
std::lock_guard l{m1,s1};
\end{lstlisting}
\end{block}

\end{frame}
