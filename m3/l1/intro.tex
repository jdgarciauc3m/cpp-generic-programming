\subsection{Constraing templates with concepts}

\begin{frame}[t,fragile]{Implicit requirements}
\begin{itemize}
  \item A generic function has a set of implied requirements.
\end{itemize}

\mode<presentation>{\vfill}
\begin{columns}[T]

\column{.5\textwidth}
\begin{block}{A generic square}
\begin{lstlisting}
template <typename T>
T square(T x) {
  return x*x;
}
\end{lstlisting}
\end{block}

\column{.5\textwidth}
\begin{itemize}
  \item Binary operator \cppkey{*} must be defined on \cppid{T}.
  \item Result must be convertible to \cppid{T}.
  \item Values of type \cppid{T} must be returnable.
\end{itemize}

\end{columns}

\mode<presentation>{\vfill\pause}
\begin{itemize}
  \item \textgood{Solution}: Define only for integral types.
\end{itemize}

\end{frame}

\begin{frame}[t,fragile]{Requiring a subset of types}
\begin{itemize}
  \item A concept from the standard library can be used.
    \begin{itemize}
      \item \cppid{std::integral} is a concept for integral types.
    \end{itemize}
\end{itemize}

\begin{block}{A generic square for integral types}
\begin{lstlisting}
std::integral auto square(std::integral auto x) {
  return x*x;
}
\end{lstlisting}
\end{block}

\mode<presentation>{\vspace{-1em}}
\begin{columns}[T]

\pause
\column{.5\textwidth}
\begin{block}{Constraints in parameter}
\begin{lstlisting}
template <std::integral T>
T square(T x) { 
  return x*x; 
}
\end{lstlisting}
\end{block}

\pause
\column{.5\textwidth}
\begin{block}{Constraints as requirements}
\begin{lstlisting}
template <typename T>
requires std::integral<T>
T square(T x) { 
  return x*x; 
}
\end{lstlisting}
\end{block}

\end{columns}

\end{frame}


\begin{frame}[t,fragile]{Defining a concept}
\begin{itemize}
  \item A \textemph{concept} on a type can be defined as a boolean expression.
    \begin{itemize}
      \item We may use any boolean expression whose value is known at compile time.
    \end{itemize}
\end{itemize}

\begin{columns}[T]

\column{.5\textwidth}
\begin{block}{An arithmetic concept}
\begin{lstlisting}
template <typename T>
concept arithmetic = 
  std::is_arithmetic_v<T>;
\end{lstlisting}
\end{block}

\column{.5\textwidth}
\begin{block}{Using the arithmetic concept}
\begin{lstlisting}
arithmetic auto square(arithmetic auto x) {
  return x * x;
}
\end{lstlisting}
\end{block}

\end{columns}
\end{frame}

\begin{frame}[t,fragile]{Overloading with concepts}
\begin{itemize}
  \item A generic function can be overloaded for sets of types.
    \begin{itemize}
      \item \textmark{Example}: Pointers versus non-pointers.
    \end{itemize}
\end{itemize}

\begin{block}{Printing values and pointers}
\begin{lstlisting}
template <typename T>
requires (not std::is_pointer_v<T>)
void print(T value) {
  std::cout << "Value: " << value << '\n';
}

template <typename T>
requires std::is_pointer_v<T>
void print(T ptr) {
  std::cout << "Pointer: " << static_cast<void*>(ptr) << ". Value: " << *ptr << '\n';
}
\end{lstlisting}
\end{block}

\end{frame}

\begin{frame}[t,fragile]{Trailing function requirements}
\begin{itemize}
  \item A requirement may be stated between function specification and function body.
    \begin{itemize}
      \item Useful if function parameters are used.
    \end{itemize}
\end{itemize}

\begin{block}{Using trailing requirements}
\begin{lstlisting}
void print(auto ptr)
requires std::is_pointer_v<decltype(ptr)>
{
  std::cout << "Pointer: " << static_cast<void*>(ptr) << ". Value: " << *ptr << '\n';
}
\end{lstlisting}
\end{block}
\end{frame}
