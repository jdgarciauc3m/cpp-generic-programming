\subsection{Requires expressions}

\begin{frame}[t,fragile]{A requires expression is not a clause}
\begin{itemize}
  \item A \textemph{requires-expression} is different from a 
       \textemph{requires-clause}.
    \begin{itemize}
      \item \textemph{requires-clause}:
            Expresses requirement on template parameters or function parameters.

      \item \textemph{requires-expression}:
            Specifies syntactic requirements on template parameters.
    \end{itemize}

\pause
\begin{columns}[T]

\column{.5\textwidth}
\begin{block}{Requires expression in concept}
\begin{lstlisting}[escapechar=@]
template <typename T>
concept incrementable = requires(T x) {
  ++x;
};
@\pause@
template <incrementable T>
T inc(T x) {
  return ++x;
}
\end{lstlisting}
\end{block}

\pause
\column{.5\textwidth}
\begin{block}{Requires expression in requires clause}
\begin{lstlisting}
template <typename T>
  requires(
    requires(T x) { 
      ++x; 
    })
T inc(T x) {
  return ++x;
}
\end{lstlisting}
\end{block}

\end{columns}

\end{itemize}
\end{frame}

\begin{frame}[t,fragile]{Requirements in requires expression}
\begin{itemize}
  \item A \textemph{requirement expression} contains:
    \pause
    \begin{itemize}
      \item A list of parameters to be used its expressions.
        \begin{itemize}
          \item Similar to function declarations.
        \end{itemize}
\begin{lstlisting}[escapechar=@]
template <typename T> 
concept incrementable = requires(@\color{red}T x@) { /*...*/ };
\end{lstlisting}

      \pause
      \item A list of requirements:
        \begin{itemize}
          \item Simple requirements.
          \item Type requirements.
          \item Compound requirements.
          \item Nested requirements.
        \end{itemize}
\begin{lstlisting}[escapechar=@]
template <typename T> 
concept incrementable = requires(T x) { 
  @\color{red}x++;@ // T supports post-increment
  //...
}
\end{lstlisting}
    \end{itemize}

\end{itemize}
\end{frame}

\begin{frame}[t,fragile]{Simple requirements}
\begin{itemize}
  \item Every \textmark{simple requirement} is an expression.
    \begin{itemize}
      \item The requirement is satisfied if the expression is valid.
      \item No execution happens at run-time.
    \end{itemize}

\begin{lstlisting}
template <typename R>
concept range = requires(R & rng) {
  begin(rng);
  end(rng);
};
\end{lstlisting}

  \mode<presentation>{\vfill\pause}
  \item A simple requirement may use more than one parameter.
\begin{lstlisting}
template <typename T>
concept addable = requires(T x, T y) {
  x+y;
  y+x;
};
\end{lstlisting}

\end{itemize}
\end{frame}

\begin{frame}[t,fragile]{Type requirements}
\begin{itemize}
  \item A \textmark{type requirement} is specified by the keyword
        \cppkey{typename} followed by a type name (it can be qualified).

  \mode<presentation>{\vfill\pause}
  \item A \textmark{type requirement} is satisfied if the named type is valid.
    \begin{itemize}
      \item Verify that a nested type exists.
      \item Verify that a class template specialization names a type.
      \item Verify that an alias template specialization names a type.
    \end{itemize}
\end{itemize}
\end{frame}

\begin{frame}[t,fragile]{Nested types requirements}
\begin{itemize}
  \item Nested types requirements must exist to be satisfied.
\begin{lstlisting}
template <typename C>
concept container = requires {
  typename C::value_type;
  typename C::size_type;
  typename C::reference;
  typename C::const_reference;
  typename C::pointer;
  typename C::const_pointer;
  typename C::iterator;
  typename C::const_iterator;
  typename C::reverse_iterator;
  typename C::const_reverse_iterator;
};
\end{lstlisting}
\end{itemize}
\end{frame}

\begin{frame}[t,fragile]{Specialization and type requirements}
\begin{itemize}
  \item Verifying if a specialization names a type.
\end{itemize}
\begin{lstlisting}
template <std::integral> class safe_value;

template <> class safe_value<int> { /* ... */ }

template <typename T>
concept safe = requires {
  typename safe_value<T>;
};

template <safe T> compute() { /*...*/ }

void f() {
  compute<int>(); // OK. safe_value<int> is valid
  compute<float>(); // Error. safe_value<float> is invalid
  compute<long>(); // OK. safe_value<int> names incomplete type
  //...
}
\end{lstlisting}
\end{frame}

\begin{frame}[t,fragile]{Specialization existence} 
\begin{itemize}
  \item Checking that an specialization really exists.
\end{itemize}
\begin{lstlisting}[escapechar=@]
template <std::integral> class safe_value;

template <> class safe_value<int> { /* ... */ }

template <typename T>
concept safe = requires {
  @\color{red}safe\_value<T>{};@
};

template <safe T> compute() { /*...*/ }

void f() {
  compute<int>(); // OK. safe_value<int> is valid.
  compute<float>(); // Error. safe_value<float> is invalid.
  @\color{red}compute<long>();@ // Error. Instantiation undefined.
  //...
}
\end{lstlisting}
\end{frame}

\begin{frame}[t,fragile]{Compound requirements}
\begin{itemize}
  \item A \textmark{compound requirement} is formed by:
    \begin{itemize}
      \item An \textmark{expression} enclosed in braces.
      \item An optional \cppkey{noexcept} specifier.
      \item An optional \textmark{requirement} on the return type.
    \end{itemize}

  \mode<presentation>{\vfill\pause}
  \item Constraint checking is performed in three steps:
    \begin{enumerate}
      \item Template arguments are substituted.
      \item If there is a \cppkey{noexcept} specifier and the expression
            \textmark{might throw}, the requirement is not satisfied.
      \item If there is a \textmark{return requirement},
            \cppkey{decltype((expression))} must satisfy such requirement.
    \end{enumerate}
\end{itemize}
\end{frame}

\begin{frame}[t,fragile]{Using compound requirements}
\begin{lstlisting}
template <typename T>
concept smart_pointer = requires(T ptr) {
  typename T::pointer;
  typename T::element_type;
  typename T::deleter_type;
  { *ptr } -> std::convertible_to<typename T::element_type>;
  { ptr.operator->() } noexcept -> std::same_as<typename T::pointer>;
  { static_cast<bool>(ptr) } noexcept -> std::same_as<bool>;
};
\end{lstlisting}

\begin{itemize}
  \item The result type is passed implicitly as first argument
        to the return requirement.
    \begin{itemize}
      \item We cannot manipulate the resulting type.
    \end{itemize}
\end{itemize}

\end{frame}

\begin{frame}[t,fragile]{Nested requirements}
\begin{itemize}
  \item A \textmark{nested requirement} includes the \cppkey{requires}
        keyword followed by a constraint expression.
    \begin{itemize}
      \item It allows to specify additional constraints in terms
            of local parameters.
    \end{itemize}

\begin{lstlisting}
template <typename T>
concept smart_pointer = requires(T ptr) {
  typename T::pointer;
  typename T::element_type;
  typename T::deleter_type;
  *ptr;
  requires std::same_as<std::remove_reference_t<decltype(*ptr)>,
      typename T::element_type>;
  { ptr.operator->() } noexcept -> std::same_as<typename T::pointer>;
  { static_cast<bool>(ptr) } noexcept -> std::same_as<bool>;
};
\end{lstlisting}
\end{itemize}
\end{frame}

\begin{frame}[t,fragile]{Type predicate versus nested requirements evaluation}
\begin{itemize}
  \item There is a difference when using type predicates.
\end{itemize}

\begin{columns}[T]

\column{.5\textwidth}
\begin{block}{Using a type predicate}
\begin{lstlisting}
template<typename T>
concept primitive_pointer = requires(T ptr) {
  std::is_pointer_v<T>;
};
\end{lstlisting}
\end{block}

\begin{itemize}
  \item Check that \cppid{is\_pointer\_v<T>} is valid.
\end{itemize}

\column{.5\textwidth}
\begin{block}{Requiring a type predicate}
\begin{lstlisting}
template<typename T>
concept primitive_pointer = requires(T ptr) {
  requires std::is_pointer_v<T>;
};
\end{lstlisting}
\end{block}

\begin{itemize}
  \item Check that \cppid{is\_pointer\_v<T>} is \cppkey{true}.
\end{itemize}

\end{columns}
\end{frame}
