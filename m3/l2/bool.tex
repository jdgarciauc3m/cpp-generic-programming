\subsection{Boolean compile-time expressions}

\begin{frame}[t,fragile]{Boolean expressions in requirements}
\begin{itemize}
  \item A \textmark{requires clause} may include any compile-time
        expression whose result is \cppkey{bool}:
    \begin{itemize}
      \mode<presentation>{\vfill}
      \item A compile-time expression that may (or not) include a type.
        \begin{itemize}
          \item Equivalent to the condition in \cppkey{static\_assert}.
        \end{itemize}

      \mode<presentation>{\vfill\pause}
      \item A compile-time variable 
            (\cppkey{constexpr} or \cppkey{constinit}).
        \begin{itemize}
          \item \cppkey{constexpr}: Initialized at compile-time 
                and cannot be modified.
          \item \cppkey{constinit}: Initialized at compile-time 
                but can be modified.
        \end{itemize}

      \mode<presentation>{\vfill\pause}
      \item A compile-time function invocation 
            (\cppkey{constexpr} or \cppkey{consteval}).
        \begin{itemize}
          \item \cppkey{constexpr}: Evaluated at compile-time if arguments
                are known at compile-time.
          \item \cppkey{consteval}: Evaluated at compile-time or
                compiler error.
        \end{itemize}
    \end{itemize}
\end{itemize}
\end{frame}


\begin{frame}[t,fragile]{Boolean requirements}
\begin{itemize}
  \item A requirement may be a non-dependent boolean expression.
\begin{lstlisting}
template <typename T>
    requires (sizeof(long) == sizeof(void*))
void f(T x);
\end{lstlisting}

  \mode<presentation>{\vfill\pause}
  \item A requirement may be a boolean expression depending on a parameter.
\begin{lstlisting}
template <typename T>
    requires (sizeof(T) <= 2 * sizeof(void*))
void f(T x);
\end{lstlisting}

  \mode<presentation>{\vfill\pause}
  \item A requirement may depend on a NTTP value.
\begin{lstlisting}
template <typename T, std::size_t N>
    requires (sizeof(T[N]) <= 4096)
class fixed_vector { /*...*/ };
\end{lstlisting}

\end{itemize}
\end{frame}

\begin{frame}[t,fragile]{Boolean requirements and compile-time variables}
\begin{itemize}
  \item Compile-time variables may be used in requirement clauses.
    \begin{itemize}
      \item All predicates in \cppid{<type\_traits>} are compile time variables.
    \end{itemize}
\begin{lstlisting}
template <typename T>
    requires (std::is_integral<T>::value)
void f(T x);

template <typename T>
    requires (std::is_integral_v<T>)
void f(T x);
\end{lstlisting}

  \mode<presentation>{\vfill\pause}
  \item A combination of expressions can also be used.
\begin{lstlisting}
template <typename T>
    requires (std::is_bounded_array_v<T> and 
              std::is_convertible_v<remove_all_extents_t<T>, std::string>)
void f(T x);
\end{lstlisting}

\end{itemize}
\end{frame}

\begin{frame}[t,fragile]{Boolean requirements and compile-time functions}
\begin{itemize}
  \item A complicated requirement can be isolated in a compile-time
        function.
\begin{lstlisting}
template <typename T>
consteval bool is_poly_reference() {
  return std::is_lvalue_reference_v<T> and
         std::is_polymorphic_v<std::remove_reference_t<T>>;
}

template <typename T>
    requires (is_poly_reference<T>())
void f(T x);
\end{lstlisting}

  \mode<presentation>{\vfill\pause}
  \item Usually, this is better expressed with a concept.
\begin{lstlisting}
template <typename T>
concept poly_reference = std::is_lvalue_reference_v<T> and
                         std::is_polymorphic_v<std::remove_reference_t<T>>;

void f(poly_reference auto x);
\end{lstlisting}
\end{itemize}
\end{frame}
