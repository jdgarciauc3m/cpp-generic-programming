\subsection{Requirement clauses}

\begin{frame}[t,fragile]{Requires clauses}
\begin{itemize}
  \item A \textemph{requires clause} may be used to constrain a template
        or to constrain a function.

\begin{columns}[T]

\pause
\column{.5\textwidth}
\begin{block}{Constraining templates}
\begin{lstlisting}
template <typename T>
requires std::integral<T>
T square(T x);

template <typename T>
requires std::semiregular<T>
class fixed_vector { /*...*/ };
\end{lstlisting}
\end{block}

\pause
\column{.5\textwidth}
\begin{block}{Constraining functions}
\begin{lstlisting}
auto square(auto x)
requires (std::integral<decltype(x)>);

template <std::semiregular T>
class fixed_vector {
  T operator[](int i) const 
  requires(sizeof(T) <= 2 * sizeof(void*));
  //...
};
\end{lstlisting}
\end{block}

\end{columns}
\end{itemize}
\end{frame}

\begin{frame}[t,fragile]{Anatomy of a requires clause}
\begin{itemize}
  \item A \textmark{requires clause} is given by the \cppkey{requires}
        keyword followed by:
    \begin{itemize}
      \item A boolean compile-time expression.
      \item A \textmark{concept}.
      \item A \textmark{require expression}.
      \item A conjunction of the above (with \cppkey{and} or \cppkey{\&\&}).
      \item A disjunction of the above (with \cppkey{or} or \cppkey{||}).
    \end{itemize}
\end{itemize}

\begin{block}{A requires clause}
\begin{lstlisting}
template <typename T>
requires std::integral<T> and
         std::is_signed_v<T> and
         requires(T x) { x*x; }
T f(T value);
\end{lstlisting}
\end{block}
\end{frame}
