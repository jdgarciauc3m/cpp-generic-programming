\subsection{Introduction to dependent names}

\begin{frame}[t,fragile]{Finding names}
\begin{itemize}
  \item Any name used in a template definition needs to be found.
    \begin{itemize}
      \item \textmark{Name lookup}: 
            Finding the declaration that introduced a name.
      \item \textmark{Name binding}: 
            Associating a name explicitly or implicitly used in a template
            with the declaration that introduced it.
    \end{itemize}

  \mode<presentation>{\vfill\pause}
  \item Names used in a template definitions can be:
    \begin{itemize}
      \item \textgood{Non-dependent names}: 
            Names that do not depend on template parameters.

      \item \textgood{Dependent names}:
            Names that depend on on a template parameter.
    \end{itemize}
\end{itemize}
\end{frame}

\begin{frame}[t,fragile]{Non-dependent names}
\begin{itemize}
  \item \textgood{Non-dependent names} are bound at the point of definition
        of the template.
\end{itemize}

\begin{block}{A min value generic function}
\begin{lstlisting}[escapechar=@]
template <typename T>
auto min_value(@\color{red}std::vector@<T> const & v) {
  if (v.empty()) { throw @\color{red}std::invalid\_argument@{"empty vector"}; }
  auto result = v[0];
  for (auto const & value : @\color{red}std::span@{v.begin()+1, v.end()}) {
    if (value < result) { result = value; }
  }
  return result;
}
\end{lstlisting}
\end{block}
\end{frame}

\begin{frame}[t,fragile]{Dependent names}
\begin{itemize}
  \item \textgood{Dependent names} are bound at the point of instantiation.
\end{itemize}
\begin{block}{A min value generic function}
\begin{lstlisting}[escapechar=@]
template <typename T>
auto min_value(std::vector<T> const & v) {
  if (@\color{red}v.empty()@) { throw std::invalid_argument{"empty vector"}; }
  auto result = @\color{red}v[0]@;
  for (auto const & value : std::span{@\color{red}v.begin()+1@, @\color{red}v.end()@}) {
    if (value @\color{red}<@ result) { result @\color{red}=@ value; }
  }
  return result;
}
\end{lstlisting}
\end{block}
\end{frame}

\begin{frame}[t,fragile]
\begin{block}{Error with point of instantiation}
\begin{lstlisting}[escapechar=@]
class string_value;

void print_min(std::vector<string_value> const & v) {
  std::cout << std::format("min = {}\n", @\color{red}min\_value(v)@.to_string()); // string_value is incomplete
}

class string_value {
public:
  explicit string_value(std::string init) : value_{std::move(init)} {}
  friend auto operator<=>(string_value const &, string_value const &) = default;
  [[nodiscard]] auto to_string() const { return value_; }
private:
  std::string value_;
};
\end{lstlisting}
\end{block}

\end{frame}


\begin{frame}[t,fragile]
\begin{block}{Correct point of instantiation}
\begin{lstlisting}
class string_value {
public:
  explicit string_value(std::string init) : value_{std::move(init)} {}
  friend auto operator<=>(string_value const &, string_value const &) = default;
  [[nodiscard]] auto to_string() const { return value_; }
private:
  std::string value_;
}; 

void print_min(std::vector<string_value> const & v) {
  std::cout << std::format("min = {}\n", min_value(v).to_string());
}

\end{lstlisting}
\end{block}

\end{frame}
