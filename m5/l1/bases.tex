\subsection{Dependent base clases}

\begin{frame}[t,fragile]{Naming and base classes}
\begin{itemize}
  \item When a template has a base class, it can access names from the base class.

  \mode<presentation>{\vfill\pause}
  \item \textmark{Alternatives}:
    \begin{itemize}
      \item The base class \textgood{does not depend} on template parameter.
      \item The base class \textbad{does depend} on template parameter.
    \end{itemize}

\pause
\begin{columns}[T]

\column{.5\textwidth}
\begin{block}{Non dependent base}
\begin{lstlisting}[basicstyle=\mode<presentation>{\tiny}]
struct shape {
  void print_origin();
  //...
};

struct circle : public shape {
  void print() {
    print_origin();
    //...
  }
  //...
};
\end{lstlisting}
\end{block}

\column{.5\textwidth}
\begin{block}{Dependent base}
\begin{lstlisting}[basicstyle=\mode<presentation>{\tiny}]
template <typename T> struct shape {
  void print_origin();
  //...
};

template <typename T> 
struct circle : public shape<T> {
  void print() {
    print_origin(); // Error: Undeclared identifier
    //...
  }
  //...
};
\end{lstlisting}
\end{block}

\end{columns}

\end{itemize}
\end{frame}

\begin{frame}[t,fragile]{}
\begin{itemize}
  \item Names from a dependent base are \textbad{not considered} in lookup.

  \mode<presentation>{\vfill\pause}
  \item \textmark{Alternatives}:
    \begin{itemize}

      \mode<presentation>{\vfill\pause}
      \item Qualify the name with a dependent type.
\begin{lstlisting}
void print() {
  shape<T>::print_origin();
  //...
}
\end{lstlisting}

      \mode<presentation>{\vfill\pause}
      \item Use an object to invoke the member.
\begin{lstlisting}
void print() {
  this->print_origin();
  //...
}
\end{lstlisting}

      \mode<presentation>{\vfill\pause}
      \item Inject the name in the scope via \cppkey{using} declaration.
\begin{lstlisting}
using shape<T>::print_origin;
void print() {
  print_origin();
  //...
}
\end{lstlisting}
    \end{itemize}
\end{itemize}
\end{frame}

\begin{frame}[t,fragile]{Dependent bases in practice}
\mode<presentation>{\vspace{-.75em}}
\begin{columns}[T]

\column{.5\textwidth}
\begin{block}{A generic shape}
\begin{lstlisting}
template <position P>
class shape {
public:
  explicit shape(P pos) : pos_{pos} {}

  void print_origin() {
    std::cout << "origin = ";
    pos_.print();
  }

private:
  P pos_;
};
\end{lstlisting}
\end{block}

\column{.5\textwidth}
\begin{block}{A generic circle}
\begin{lstlisting}
template <position P>
class circle : public shape<P> {
public:
  circle(P pos, double r) : shape<P>{pos}, radius{r} {}

  void print()  {
    this->print_origin();
    std::cout << std::format(" radius = {}", radius);
  }

private:
  double radius;
};
\end{lstlisting}
\end{block}

\end{columns}
\end{frame}
