\subsection{Templates as parameters}

\begin{frame}[t,fragile]{Passing and returning containers}
\mode<presentation>{\vspace{-.75em}}
\begin{itemize}
  \item Consider a function \cppid{convert()} to copy elements from one container
        to a different container:
\begin{lstlisting}
std::vector vec{1.5, 2.5, 3.5};
auto lst = convert<std::list<double>>(vec);
auto deq = convert<std::deque<double>>(vec);
\end{lstlisting}

  \mode<presentation>{\vfill\pause}
  \item Must take two template parameters for the input (\cppid{T}) and the return (\cppid{R}).
\begin{block}{A first convert function}
\begin{lstlisting}
template<typename R, typename T>
R convert(T const & cont) {
  R result;
  std::ranges::copy(cont, std::back_inserter(result));
  return result;
}
\end{lstlisting}
\end{block}
  
\end{itemize}
\end{frame}

\begin{frame}[t,fragile]{Using template template parameters}
\begin{itemize}
  \item A \cppkey{template template} parameter is not a type, but a template.
\begin{block}{A convert function with a template template parameter}
\begin{lstlisting}
template<template<typename, typename> typename R,
    template<typename, typename> typename T, typename E, typename A>
auto convert(T<E, A> const &cont) {
  R<E, A> result;
  std::ranges::copy(cont, std::back_inserter(result));
  return result;
}
\end{lstlisting}
\end{block}

  \mode<presentation>{\vfill\pause}
  \item When invoked explicit parameters are not types, but templates.
\begin{lstlisting}
  auto lst = convert<std::list>(vec);
\end{lstlisting}
\end{itemize}
\end{frame}

