\subsection{Tuple-like interface for arrays}

\begin{frame}[t,fragile]{Compile time indexing}
\begin{itemize}
  \item Global function \cppid{std::get<I>(a)} gets 
        I\textsuperscript{th} element of array \cppid{a}.
    \begin{itemize}
      \item \cppid{I} must be known at compile-time.
      \item \cppid{I} must be in the range \cppid{[0,N)}.
    \end{itemize}
\begin{lstlisting}
std::array vec{ 1, 2, 3, 4, 5 };
auto x = std::get<2>(vec); // Equivalent to x = vec[2]
std::get<0>(vec) = 42;
\end{lstlisting}

  \mode<presentation>{\vfill\pause}
  \item \textbad{Note}: The index must be known at compile time.
    \begin{itemize}
      \item A regular loop is not possible.
      \item But an implementation using \cppid{std::index\_sequence} is possible.
    \end{itemize}

\end{itemize}
\end{frame}

\begin{frame}[t,fragile]{Tuple-like metafunctions}
\begin{itemize}
  \item \cppid{std::tuple\_size\_v<A>}: Size of the array type \cppid{A}.
    \begin{itemize}
      \item The value is a compile-time constant.
    \end{itemize}
\begin{lstlisting}
std::array vec{1,2,3};
constexpr auto sz1 = vec.size(); // OK
auto & vec2 = vec;
constexpr auto sz2 = vec.size(); // Error
constexpr auto sz3 = std::tuple_size_v<std::decay_t<decltype(vec2)>>; // OK
\end{lstlisting}

  \item \cppid{std::tuple\_element\_t<I,A>}: Type of I\textsuperscript{th}
        element of the array.
    \begin{itemize}
      \item For compatibility with \cppid{std::tuple}.
      \item All elements have the same type.
    \end{itemize}
\begin{lstlisting}
std::array vec{1,2,3};
using basetype = std::tuple_element_t<0,decltype(vec)>;
\end{lstlisting}
\end{itemize}
\end{frame}
