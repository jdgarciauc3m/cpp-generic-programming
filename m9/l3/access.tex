\subsection{Accessing elements}

\begin{frame}[t,fragile]{Element access in a tuple}
  \begin{itemize}
    \item Non-member function \cppid{get<N>(t)}.
\begin{lstlisting}
std::tuple<std::string,int,std::map<std::string,int>> e;
auto name = std::get<0>(e); // string
auto age = std::get<1>(e); // int
auto marks = std::get<2>(e);  // map<string,int>
\end{lstlisting}

    \mode<presentation>{\vfill}
    \item Access index must be known at compile time.
      \begin{itemize}
        \item It is not possible to write loops for tuples.
      \end{itemize}
  \end{itemize}
\end{frame}

\begin{frame}[t,fragile]{Other form of tuple access}
\begin{itemize}
  \item The elements of a tuple can be accessed by type.
    \begin{itemize}
      \item In addition to index.
    \end{itemize}
\end{itemize}
\begin{lstlisting}
std::tuple<string, string, int> a { "Daniel", "Garcia", 1969 };
auto x = std::get<2>(a); // int x = 1969
auto y = std::get<int>(a); // int x = 1969
auto z = std::get<string>(a) // Error.
\end{lstlisting}
\end{frame}

\begin{frame}[t,fragile]{Tuple and returning multiple values from functions}
  \begin{itemize}
    \item A tuple can be used to return multiple values.
\begin{lstlisting}
std::tuple<int,int> divide(int x, int y) {
  return std::make_tuple(x/y, x%y);
}

auto r = divide(10,3);
int quotient = std::get<0>(r);
int remainder = std::get<1>(r);
\end{lstlisting}

    \mode<presentation>{\vfill\pause}
    \item \cppid{tie} provides a way to associate tuple elements
          with variables.
\begin{lstlisting}
int quotient, remainder;
std::tie(quotient, remainder) = divide(10,3);
std::tie(quotient, std::ignore) = divide(15,2);
std::tie(std::ignore, remainder) = divide(3, 2);
\end{lstlisting}

    \mode<presentation>{\vfill}
    \item Solution for functions with multiple return values.
  \end{itemize}
\end{frame}

\begin{frame}[t,fragile]{Simplifying multiple returns with structured bindings}
\begin{itemize}
  \item Structured bindings offer a simpler way of decomposing
        the value of a \cppid{std::tuple}.
\begin{lstlisting}
auto [q1, r1] = divide(10,3);
\end{lstlisting}
\end{itemize}
\end{frame}

\begin{frame}[t,fragile]{Ask your tuple (at compile time)}
  \begin{itemize}
    \item You can get the size of tuple and the types of its elements.
\begin{lstlisting}
using record = std::tuple<std::string, int, std::vector<int>>;
using entry = std::tuple<int, record>

template <typename T>
void f(T t) {
  constexpr std::size_t num = std::tuple_size_v<T>;

  using key_type = typename std::tuple_element_t<0, T>;

  // ...
}
\end{lstlisting}
  \end{itemize}
\end{frame}

