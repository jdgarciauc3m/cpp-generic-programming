\subsection{A n-dimensional tuple}

\begin{frame}[t,fragile]{What's a tuple?}
  \begin{itemize}
    \item \cppid{std::tuple}: 
          A natural extension of the \cppid{std::pair} using 
          \textmark{variadic templates}.
\begin{lstlisting}
template <typename ... T>
class tuple;
\end{lstlisting}

    \mode<presentation>{\pause\vfill}
    \item A tuple is a sequence of \cppid{N} elements of arbitrary types.
      \begin{itemize}
        \item Type for each element is known at compile time.
\begin{lstlisting}
std::tuple<string, int, double> student{"Daniel", 1965, 1.82};
\end{lstlisting}
        \item Its size is known at compile time.
        \item It does not directly use dynamic memory.
      \end{itemize}
  \end{itemize}
\end{frame}

\begin{frame}[t,fragile]{Creaing tuples}
  \begin{itemize}
    \item \cppid{std::tuple} has multiple constructors:
      \begin{itemize}
        \item Deduction guides usally help.
\begin{lstlisting}
std::tuple student{"Daniel"s, 1965, 1.82}; // tuple<std::string, int, double>
std::tuple other{"Carlos", 2003, 1.96}; // tuple<char const*, int, double> 
\end{lstlisting}

        \item There is also a \cppid{std::make\_tuple()} function.
\begin{lstlisting}
template <typename ... T>
std::tuple <T...> make_tuple(T&& ... t) {
  return std::tuple<T...>{t...};
}
\end{lstlisting}
      \item Specially convenient in combination with \cppid{auto}.
\begin{lstlisting}
auto t = std::make_tuple(std::string("Carlos"), 2003, 1.96);
\end{lstlisting}
  \end{itemize}
\end{itemize}
\end{frame}
