\chapter*{Foreword}

Welcome!

Generic programming has become one of the cornerstones of C++ over the years.

This is not a book. It is just the support material for a course on how you can
make use of generic programming in modern C++. 
Note that this course assumes that you
already have enough background on the C++ programming language.

I hope you find this course useful to improve your skills as a software
engineer allowing you to write better software.

\section*{Structure}

The contents of this course is organized in a set of topics. 
Refer to the table of contents for details.

\section*{Work in progress contents}

Below you will find a list of topics that are not currently included int the materials:

\begin{itemize}
  \item Some standard library utilites:
    \begin{itemize}
      \item \cppid{std::array}.
      \item \cppid{std::span}.
      \item \cppid{std::tuple}.
      \item \cppid{std::variant}.
      \item \cppid{std::any}
    \end{itemize}
  \item Compile-time computations:
    \begin{itemize}
      \item \cppkey{consteval} and \cppkey{constinit}.
      \item Constant evaluated contexts.
    \end{itemize}
  \item Lambda extensions from C++20:
    \begin{itemize}
      \item Default constructor in lambdas.
      \item Lambdas as NTTPs.
      \item Lambdas with \cppkey{consteval}.
      \item Captures \cppkey{[=,this]} and \cppkey{[*this]}.
      \item Capture of structured bindings.
      \item Variadic captures and init capture.
    \end{itemize}
  \item Advanced techniques:
    \begin{itemize}
      \item Type erasure.
      \item CRTP.
    \end{itemize}
\end{itemize}
