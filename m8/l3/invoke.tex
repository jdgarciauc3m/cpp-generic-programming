\subsection{An interface for callable invocations}

\begin{frame}[t,fragile]{Invoking member functions}
\begin{block}{Measuring a member function}
\begin{lstlisting}[escapechar=@]
void test() {
  map region{make_coordinates(1000)};                              
  auto [dif2, distance] = measure(&map::distance_to, region, 1.5, 1, .5);  
  std::cout << std::format("Distance = {:.2}\n", distance);
  std::cout << std::format("Time to compute = {:L}\n", dif2);
}
\end{lstlisting}
\end{block}
\begin{itemize}
  \item \textbad{Problem}: \cppid{\&map::distance\_to} is member function.
    \begin{itemize}
      \item \cppid{fun(args...)} is not a valid syntax for member functions.
      \item Instead \cppid{arg0.fun(other\_args...)}.
    \end{itemize}
\end{itemize}
\end{frame}

\begin{frame}[t,fragile]{Generalized invocations}
\begin{itemize}
  \item Type \cppid{invoke\_result<F, Args...>}: 
        Deduces the return type of invoking \cppid{F} with \cppid{Args...}.
    \begin{itemize}
      \item If the invocation is valid the return type is the
            nested type \cppid{type}.
    \end{itemize}

  \mode<presentation>{\vfill\pause}
  \item \cppid{std::invoke()} can be used to invoke a callable.
\begin{lstlisting}
template <typename F, typename ... Args>
std::invoke_result_t<F,Args...> invoke(F && f, Args && ... args);
\end{lstlisting}
\end{itemize}
\end{frame}

\begin{frame}[t,fragile]{Invocation of a function}
\begin{itemize}
  \item For functions or function pointers:
    \begin{itemize}
      \item \cppid{std::invoke(fun, arg1, arg2, ..., argN)}
            is equivalent to 
            \cppid{fun(arg1, arg2, ..., argN)}
      \item \cppid{std::invoke\_result<F,A1,A2, ..., AN>}
            is the result type of such invocation.
    \end{itemize}
\end{itemize}
\begin{block}{Function invocation}
\begin{lstlisting}
auto positive(int x) { return x>0; }

void f() {
  using ret1 = std::invoke_result_t<decltype(positive), int>;
  static_assert(std::is_same_v<ret1, bool>);

  auto res1 = std::invoke(positive, 42);
  std::cout << std::format("res1 = {}", res1);
}
\end{lstlisting}
\end{block}
\end{frame}

\begin{frame}[t,fragile]{Invocation of a function object}
\begin{itemize}
  \item For functions object types:
    \begin{itemize}
      \item \cppid{std::invoke(funobj, arg1, arg2, ..., argN)}
            is equivalent to 
            \cppid{funobj::operator()(arg1, arg2, ..., argN)}
      \item \cppid{std::invoke\_result<F,A1,A2, ..., AN>}
            is the result type of such invocation.
    \end{itemize}
\end{itemize}
\begin{block}{Function invocation}
\begin{lstlisting}[basicstyle=\tiny]
struct max_value {
    auto operator()(double x, double y) const { return x>y?x:y; }
};

void f() {
  using ret2 = std::invoke_result_t<max_value, double, double>;
  static_assert(std::is_same_v<ret2,double>);

  max_value mvobj;
  auto res2 = std::invoke(mvobj, 1.5, 2.5);
  std::cout << std::format("res1 = {}\n", res2);
}
\end{lstlisting}
\end{block}
\end{frame}

\begin{frame}[t,fragile]{Invocation of a lambda}
\begin{itemize}
  \item For a lambda:
    \begin{itemize}
      \item \cppid{std::invoke(lambda, arg1, arg2, ..., argN)}
            is equivalent to 
            \cppid{lambda(arg1, arg2, ..., argN)}
      \item \cppid{std::invoke\_result<lambda\_type,A1,A2, ..., AN>}
            is the result type of such invocation.
    \end{itemize}
\end{itemize}
\begin{block}{Function invocation}
\begin{lstlisting}
void f() {
  auto lmb = [] { std::cout << "lambda\n"; };
  using ret3 = std::invoke_result_t<decltype(lmb)>;
  static_assert(std::is_same_v<ret3,void>);

  std::invoke(lmb);
}
\end{lstlisting}
\end{block}
\end{frame}

\begin{frame}[t,fragile]{Invocation of a function object convertible to function pointer}
\begin{itemize}
  \item For functions object types with conversion to function pointer:
    \begin{itemize}
      \item \cppid{std::invoke(funobj, arg1, arg2, ..., argN)}
            is equivalent to 
            \cppid{funptr(arg1, arg2, ..., argN)} where
            \cppid{funptr} is obtained by function pointer conversion operator.
      \item \cppid{std::invoke\_result<F,A1,A2, ..., AN>}
            is the result type of such invocation.
    \end{itemize}
\end{itemize}
\begin{block}{Function invocation}
\begin{lstlisting}[basicstyle=\tiny]
struct max_value {
    auto operator()(double x, double y) const { return x>y?x:y; }
};

void f() {
  using ret2 = std::invoke_result_t<max_value, double, double>;
  static_assert(std::is_same_v<ret2,double>);

  max_value mvobj;
  auto res2 = std::invoke(mvobj, 1.5, 2.5);
  std::cout << std::format("res1 = {}\n", res2);
}
\end{lstlisting}
\end{block}
\end{frame}

\begin{frame}[t,fragile]{Invocation of member function pointers}
\begin{itemize}
  \item For pointers to member functions of class \cppid{C}:
    \begin{itemize}
      \item For \cppid{std::invoke(f, obj, a1, ..., aN)}:
        \begin{itemize}
          \item \cppid{T0} is the unqualified type of \cppid{obj}.
          \item If \cppid{T0} is class \cppid{C} or any class derived from \cppid{C},
                then call \cppid{obj.*f(a1,...,aN)} is invoked.
          \item If \cppid{T0} is an specialization of \cppid{std::reference\_wrapper}
                then call \cppid{obj.get().*f(a1,...,aN)} is invoked.
          \item Otherwise,
                call \cppid{(*obj).*f(a1,...,aN)}
        \end{itemize}
       \item \cppid{std::invoke\_result\_t<F,T0,A1,...,AN>} 
             is the result type of such invocation.
    \end{itemize}
\end{itemize}
\begin{block}{Member function pointer invocation}
\mode<presentation>{\vspace{-1em}}
\begin{columns}[T]

\column{.45\textwidth}
\begin{lstlisting}[basicstyle=\tiny]
struct value {
    [[nodiscard]] bool greater_than(float k) const { return x>k; }
    float x;
};
\end{lstlisting}

\column{.5\textwidth}
\begin{lstlisting}[basicstyle=\tiny]
void f() {
  using ret5 = std::invoke_result_t<decltype(&value::greater_than),value,double>;
  static_assert(std::is_same_v<ret5, bool>);

  auto res5 = std::invoke(&value::greater_than, value{}, -3);
  std::cout << std::format("res5 = {}\n", res5);
}
\end{lstlisting}
\end{columns}
\end{block}
\end{frame}


\begin{frame}[t,fragile]{Invocation of member object pointers}
\begin{itemize}
  \item For pointers to member objects of class \cppid{C}:
    \begin{itemize}
      \item For \cppid{std::invoke(m, obj)}:
        \begin{itemize}
          \item \cppid{T0} is the unqualified type of \cppid{obj}.
          \item If \cppid{T0} is class \cppid{C} or any class derived from \cppid{C},
                then access \cppid{obj.*m} is performed.
          \item If \cppid{T0} is an specialization of \cppid{std::reference\_wrapper}
                then access \cppid{obj.get().*m} is performed.
          \item Otherwise,
                access \cppid{(*obj).*m}
        \end{itemize}
       \item \cppid{std::invoke\_result\_t<M,T0>} 
             is the result type of such access.
    \end{itemize}
\end{itemize}
\begin{block}{Member function pointer invocation}
\mode<presentation>{\vspace{-1em}}
\begin{columns}[T]

\column{.45\textwidth}
\begin{lstlisting}[basicstyle=\tiny]
struct value {
    [[nodiscard]] bool greater_than(float k) const { return x>k; }
    float x;
};
\end{lstlisting}

\column{.5\textwidth}
\begin{lstlisting}[basicstyle=\tiny]
void f() {
  using ret6 = std::invoke_result_t<decltype(&value::x), value>;
  static_assert(std::is_same_v<ret6,float&&>);

  value val6{33};
  auto res6 = std::invoke(&value::x, val6);
  std::cout << std::format("res6 = {}\n", res6);
}
\end{lstlisting}
\end{columns}
\end{block}
\end{frame}
