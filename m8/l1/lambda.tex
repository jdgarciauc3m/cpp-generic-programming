\subsection{Lambda expressions}

\begin{frame}[t,fragile]{What is a lambda expression?}
\mode<presentation>{\vspace{-1em}}
\begin{lstlisting}
void f() {
  std::vector v{1, -1, 2, 3, 4};
  int limit = 1;
  auto cnt = std::find_if(v, [limit](auto x) -> bool { return x>limit; });
  //...
}
\end{lstlisting}
\mode<presentation>{\vspace{-1em}}
\begin{itemize}
  \mode<presentation>{\vfill\pause}
  \item \textmark{Lambda capture}: Marks the begining of the lambda and introduces
        captured objects.
\begin{lstlisting}[escapechar=@]
@\color{red}[limit]@(auto x) -> bool { return x>limit; }
\end{lstlisting}

  \mode<presentation>{\vfill\pause}
  \item \textmark{Lambda parameters}: Equivalent to function parameters
\begin{lstlisting}[escapechar=@]
[limit]@\color{red}(auto x)@ -> bool { return x>limit; }
\end{lstlisting}

  \mode<presentation>{\vfill\pause}
  \item \textmark{Lambda body}: Equivalent to function body
\begin{lstlisting}[escapechar=@]
[limit](auto x) -> bool @\color{red}{ return x>limit; }@
\end{lstlisting}

  \mode<presentation>{\vfill\pause}
  \item \textmark{Return type}: Usually deduced.
\begin{lstlisting}[escapechar=@]
[limit](auto x) @\color{red}-> bool@ { return x>limit; }
\end{lstlisting}

\end{itemize}
\end{frame}

\begin{frame}[t,fragile]{Lambda expressions and types}
\begin{itemize}
  \item Every \textemph{lambda expression} generates a \textmark{unique}
        \textgood{compiler generated type}.
    \begin{itemize}
      \item You \textbad{cannot} know the value of such type.
      \item However, its value can be stored in one object by using \cppkey{autor}.
\begin{lstlisting}
auto positive = [](std::integral auto x) { return x>0; };
auto cnt = std::find_if(v, positive);
\end{lstlisting}
    \end{itemize}

  \mode<presentation>{\vfill\pause}
  \item Each \textemph{lambda expression} results in a \textmark{distinct type}:
\begin{lstlisting}
auto pos = [](std::integral auto x) { return x>0; };
auto gt0 = [](std::integral auto x) { return x>0; };
static_assert(not std::is_same_v<decltype(pos), decltype(gt0)>);
\end{lstlisting}

  \mode<presentation>{\vfill\pause}
  \item The \textemph{type} is created in the \textmark{smallest enclosing scope}.

\end{itemize}
\end{frame}


\begin{frame}[t,fragile]{Closure object}
\begin{itemize}
  \item The \textmark{evaluation} of a \textemph{lambda expression} generates
        a temporal object called \textgood{closure object}.
    \begin{itemize}
      \item A \textemph{closure object} behaves as a function object.
      \item The type of a \textemph{lambda expression} is the same than 
            the type of the \textgood{closure object}.
    \end{itemize}

  \mode<presentation>{\vfill\pause}
  \item A \textemph{closure object} has a type that is:
    \begin{itemize}
      \item Unique.
      \item Unnamed.
      \item A class type that is not a union.
      \item A non-aggregate type.
      \item Declared in the smallest enclosing scope.
    \end{itemize}

\begin{lstlisting}
auto comp = [](record const & x, record const & y) { return x.id < y.id; };
std::ranges::sort(vec, comp);
bool less = comp(r1,r2);    
\end{lstlisting}
\end{itemize}
\end{frame}

\begin{frame}[t,fragile]{Closure types generated members}
\begin{itemize}
  \item A \textemph{closure type}:
    \begin{itemize}
      \item It has a \textbad{deleted} \textgood{default constructor}. 
      \item It has a \textbad{deleted} \textgood{copy assignment operator}.
      \item It has an \textmark{implicitly declared} \textgood{copy constructor}.
      \item It may have an \textmark{implicitly declared} \textgood{move constructor}.
      \item It has an \textmark{implicitly declared} \textgood{destructor}.
      \item It has a \textgood{function call operator} 
            (\cppkey{operator()}) which is \cppkey{inline}.
        \begin{itemize}
          \item Parameters and return type are the same than those in the lambda.
          \item The operator is \cppkey{const} if the lambda was declared without \cppkey{mutable}.
        \end{itemize}
    \end{itemize}
\end{itemize}
\end{frame}

\begin{frame}[t,fragile]{Conversion to function pointer}
  \begin{itemize}
    \item \textmark{Condition}:
      \begin{itemize}
        \item It must have an \textgood{empty capture}.
      \end{itemize}
    \item Generated \textemph{conversion operator}:
      \begin{itemize}
        \item Conversion to \textgood{function pointer}.
        \item \textgood{Conversion operator} \textmark{public}, \textmark{non-virtual}, 
              \textmark{non-explicit}, and \textmark{constant}.
        \item \textmark{Same} \textgood{parameters} and \textgood{return type}.
        \item Returns a \textgood{function pointer} whose invocation has the
              \textmark{same effect} than invoking the
              \textgood{function call operator} from the \textgood{closure object}.
      \end{itemize}
  \end{itemize}
\begin{lstlisting}
void set_func(void (*f)());
set_func([]{ std::cerr << "Bye"; });
\end{lstlisting}
\end{frame}

\begin{frame}[t,fragile]{Polymorphic lambdas}
\begin{itemize}
  \item It allows writing a lambda expression where one or more parameters are \textgood{generic}.
    \begin{itemize}
      \item Use \cppkey{auto}.
      \item Use a \cppkey{concept} followed by \cppkey{auto}.
    \end{itemize}
\end{itemize}
\pause
\begin{lstlisting}
std::sort(v.begin(), v.end(), [](auto const & x, auto const & y) {
  return x < y;
});
\end{lstlisting}

\mode<presentation>{\vfill\pause}
\begin{itemize}
  \item Generates a class with a \textmark{templated} \textgood{call operator}.
\end{itemize}
\begin{lstlisting}
class compiler_generated {
public:
  template <typename T1, typename T2>
  bool operator()(T1 const & x, T2 const & y) const {
    return x < y;
  }
  //...
};
\end{lstlisting}
\end{frame}


\begin{frame}[t,fragile]{Example: Printing a container}
\begin{lstlisting}
template <std::ranges::range R>
void print(R && r) {
  std::ranges::for_each(std::forward<R>(r),
    [](auto const & x) {
      std::cout << x << ' ';
    });
  std::cout << '\n';
}

void test() {
  std::vector<int> vals{1, 2, 3, 4};
  print(vals);

  std::vector<std::string> names{"Daniel", "Carlos", "Maria"};
  print(names);
}
\end{lstlisting}
\end{frame}
\begin{frame}[t,fragile]{Explcit template paramters in lambdas}
\begin{itemize}
  \item The template parameters in a lambda are normally deduced.
\begin{lstlisting}
std::sort(v.begin(), v.end(), [](auto const & x, auto const & y) {
  return x < y;
});
\end{lstlisting}

  \mode<presentation>{\vfill\pause}
  \item However, they can also be explictly specified:
\begin{lstlisting}
std::sort(v.begin(), v.end(), []<typename T>(T const & x, T const & y) {
  return x < y;
});
\end{lstlisting}
\end{itemize}
\end{frame}

