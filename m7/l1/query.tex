\subsection{Property types queries}

\begin{frame}[t,fragile]{Queries on properties of a type}
\begin{itemize}
  \item A \textmark{property query} obtains an integral property 
        on a type.
  
  \mode<presentation>{\vfill\pause}
  \item Base characteristic is \cppid{std::integral\_constant<std::size\_t, value>}.

  \mode<presentation>{\vfill\pause}
  \item Properties:
    \begin{itemize}
      \item \cppid{alignment\_of<T>}: Equivalent to \cppkey{alignof(T)}.
      \item \cppid{rank<T>}: Number of dimensions of array type.
      \item \cppid{extent<T,N>}: Number of elements for Nth dimension of array type.
      \item \cppid{extent<T>}: Number of elements for first dimension of array type.
    \end{itemize}
\end{itemize}
\end{frame}

\begin{frame}[t,fragile]
\begin{block}{Printing sizes of arrays}
\begin{lstlisting}[escapechar=@]
template <typename T, std::size_t ... I>
void print_dimension(T & , std::index_sequence<I...>) { 
  ((std::cout << std::format("  Dimension {}: {}\n",I, @\color{red}std::extent\_v<T,I>@)), ...);
}

template <typename T>
void print_info(std::string_view name, T & arr) {
  std::cout << std::format("name {}\n", name);
  if constexpr @\color{red}(std::is\_array\_v<T>)@ {
    std::cout << std::format("  is an array of {} dimensions\n", @\color{red}std::rank\_v<T>@);
    print_dimension(arr, std::make_index_sequence<@\color{red}std::rank\_v<T>@>{});
  } else {
    std::cout << "  is not an array\n";
  }
}
\end{lstlisting}
\end{block}
\end{frame}
