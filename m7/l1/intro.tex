\subsection{Introduction to metafunctions}

\begin{frame}[t,fragile]{What is a metafunction?}
\begin{itemize}
  \item A \textgood{metafunction} is a function that:
    \begin{itemize}
      \item At least \textmark{takes} one parameter which is a \textemph{type}; and/or
      \item At least \textmark{returns} one result that is a \textemph{type}.
    \end{itemize}

  \mode<presentation>{\vfill\pause}
  \item \textmark{Example}: \cppkey{sizeof}.
    \begin{itemize}
      \item Takes a type and returns an integer value.
    \end{itemize}

  \mode<presentation>{\vfill\pause}
  \item Examples in the \textgood{standard library}:
\begin{lstlisting}
static_assert(std::is_arithmetic<T>::value, "arithmetic type expected");
static_assert(std::is_arithmetic_v<T>, "arithmetic type expected");

enum class color : std::int8_t { red, green, blue };
using base_color = typename std::underlying_type<color>::type;
using base_color = std::underlying_type_t<color>;
\end{lstlisting}

\end{itemize}
\end{frame}

\begin{frame}[t,fragile]{Implementing metafunctions}
\begin{itemize}
  \item A metafunction may be implemented using a \textmark{struct template}.
\begin{lstlisting}
template <typename T, std::size_t N>
struct array_size {
    using type = T[N];
    static constexpr std::size_t size = sizeof(type);
};

using string_array_type = typename array_size<std::string,10>::type;
std::cout << "size = " << array_size<std::string,10>::size << '\n';
\end{lstlisting}

  \mode<presentation>{\vfill\pause}
  \item A metafunction may:
    \begin{itemize}
      \item Take multiple type as arguments.
      \item Take multiple values as arguments.
      \item Return multiple types as results.
      \item Return multiple values as results.
    \end{itemize}
\end{itemize}
\end{frame}

\begin{frame}[t,fragile]{Conventions in metafunctions}
\begin{itemize}
  \item Usual convenitons to simplify metafunctions use:
    \begin{itemize}
      \item A metafunction that returns a type has a nested type alias named \cppid{type}.
\begin{lstlisting}
template <typename T>
struct buffer_type {
    using type = std::byte[sizeof(T)];
};
\end{lstlisting}

      \item A metafunction that returns a value has a nested constant named \cppid{value}.
\begin{lstlisting}
template <typename T>
struct elements {
    static constexpr int value = 1;
};

template <typename T, std::size_t N>
struct elements<T[N]> {
    static constexpr int value = N;
};
\end{lstlisting}

    \end{itemize}

\end{itemize}
\end{frame}

\begin{frame}[t,fragile]{Simplifying type metafunction use}
\begin{itemize}
  \item For metafunctions returning a type an alias template can be easily defined.
\begin{lstlisting}
template <typename T>
struct buffer_type { using type = std::byte[sizeof(T)]; };

template <typename T>
using buffer_type_t = typename buffer_type<T>::type;
\end{lstlisting}
\end{itemize}

\begin{columns}[T]

\column{.5\textwidth}
\begin{block}{Without alias template}
\begin{lstlisting}[basicstyle=\tiny,escapechar=@]
template <typename T>
void print_as_buffer(T value) {
  @\color{red}typename buffer\_type<T>::type@ buffer;
  std::memcpy(&buffer, &value, sizeof(T));
  for (auto b : buffer) {
    std::cout << static_cast<int>(b) << ' ';
  }
  std::cout << '\n';
}
\end{lstlisting}
\end{block}

\column{.5\textwidth}
\begin{block}{With alias template}
\begin{lstlisting}[basicstyle=\tiny,escapechar=@]
template <typename T>
void print_as_buffer(T value) {
  @\color{red}buffer\_type\_t<T>@ buffer;
  std::memcpy(&buffer, &value, sizeof(T));
  for (auto b : buffer) {
    std::cout << static_cast<int>(b) << ' ';
  }
  std::cout << '\n';
}
\end{lstlisting}
\end{block}

\end{columns}
\end{frame}

\begin{frame}[t,fragile]{Integral constants in the standard library}
\begin{itemize}
  \item \cppid{std::integral\_costant<T,v>} wraps a static constant of the integral type \cppid{T}.
    \begin{itemize}
      \item Nested type \cppid{value\_type} is type \cppid{T}.
      \item Nested type \cppid{type} is type \cppid{std::integral\_costant<T,v>}.
      \item Constant \cppid{value} holds the value \cppid{v}.
      \item Value can be obtained by conversion to \cppid{value\_type} or by invoking \cppkey{operator()}.
    \end{itemize}

\begin{lstlisting}
  using max_type = std::integral_constant<int,100>; 
  max_type max;
  static_assert(std::is_same_v<int, max_type::value_type>);
  static_assert(std::is_same_v<max_type, max_type::type>);
  std::cout << std::format("max value {}\n", max_type::value);
  std::cout << std::format("max value {}\n", static_cast<int>(max));
  std::cout << std::format("max value {}\n", max());
  std::cout << std::format("max value {}\n", static_cast<int>(max_type{}));
  std::cout << std::format("max value {}\n", max_type{}());
\end{lstlisting}

\end{itemize}
\end{frame}

\begin{frame}[t,fragile]

\mode<presentation>{\vspace{-1em}}
\begin{columns}[T]

\column{.5\textwidth}
\begin{block}{Implementation: Basic definition}
\begin{lstlisting}[basicstyle=\tiny]
template <typename T>
struct elements {
    static constexpr int value = 1;
};

template <typename T, std::size_t N>
struct elements<T[N]> {
    static constexpr int value = N;
};

template <typename T>
static constexpr inline int elements_v = elements<T>::value;
\end{lstlisting}
\end{block}

\pause
\column{.5\textwidth}
\begin{block}{Implementation: Improved definition}
\begin{lstlisting}[basicstyle=\tiny]
template <typename T>
struct elements2 : public std::integral_constant<int, 1> {};

template <typename T, std::size_t N>
struct elements2<T[N]> : public std::integral_constant<int,N> {};

template <typename T>
static constexpr inline int elements2_v = elements2<T>::value;

\end{lstlisting}
\end{block}

\end{columns}

\pause
\begin{block}{Usage}
\begin{lstlisting}[basicstyle=\tiny]
void f() {
  std::cout << elements<int>::value;
  std::cout << elements<int[4]>::value;

  std::cout << elements_v<int>;
  std::cout << elements_v<int[4]>;
}
\end{lstlisting}
\end{block}

\end{frame}

\begin{frame}[t,fragile]{Boolean constants in the standard library}
\begin{itemize}
  \item \cppid{std::bool\_constant<v>} covers the common case of boolean constants.
    \begin{itemize}
      \item \cppid{std::integral\_constant<bool,v>}.
    \end{itemize}

  \mode<presentation>{\vfill\pause}
  \item Two type alias are also defined:
    \begin{itemize}
      \item \cppid{std::true\_type} $\Rightarrow$ \cppid{std::bool\_constant<bool,}\cppkey{true}\cppid{>}.
      \item \cppid{std::false\_type} $\Rightarrow$ \cppid{std::bool\_constant<bool,}\cppkey{false}\cppid{>}.
    \end{itemize}

\mode<presentation>{\vfill\pause}
\begin{block}{A predicate for large size types}
\begin{lstlisting}[basicstyle=\tiny]
static constexpr inline int size_limit = 128;

template <typename T>
struct is_large_type : public std::bool_constant<(sizeof(T) > size_limit)> {};

template <typename T>
static constexpr inline bool is_large_type_v = is_large_type<T>::value;

void f() {
  std::cout << std::format("int -> {}\n",is_large_type_v<int>);
  std::cout << std::format("int[100] -> {}\n",is_large_type_v<int[100]>); 
}
\end{lstlisting}
\end{block}
 
\end{itemize}
\end{frame}
