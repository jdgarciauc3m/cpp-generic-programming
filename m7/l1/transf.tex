\subsection{Type transformations}

\begin{frame}[t,fragile]{Basic transformations}
\begin{itemize}
  \item Sign transformations:
    \begin{itemize}
      \item \cppid{make\_signed<T>} / \cppid{make\_unsigned<T>}.
    \end{itemize}
  \item \cppkey{const}/\cppkey{volatile} transformations:
    \begin{itemize}
      \item \cppid{remove\_const<T>} / \cppid{add\_const<T>}.
      \item \cppid{remove\_volatile<T>} / \cppid{add\_volatile<T>}.
      \item \cppid{remove\_cv<T>} / \cppid{add\_cv<T>}.
    \end{itemize}
  \item References transformations:
    \begin{itemize}
      \item \cppid{add\_lvalue\_reference<T>}.
      \item \cppid{add\_rvalue\_reference<T>}.
      \item \cppid{remove\_reference<T>}.
    \end{itemize}
  \item Array transformations:
    \begin{itemize}
      \item \cppid{remove\_extent<T>} / \cppid{remove\_all\_extents<T>}.
    \end{itemize}
  \item Pointer transformations:
    \begin{itemize}
      \item \cppid{remove\_pointer<T>} / \cppid{add\_pointer<T>}.
    \end{itemize}
\end{itemize}
\end{frame}

\begin{frame}[t,fragile]{Other transformations}
\begin{itemize}
  \item Language support transformations:
    \begin{itemize}
      \item \cppid{underlying\_type<E>}.
      \item \cppid{decay<T>}.
      \item \cppid{invoke\_result<F,Arg...>}.
    \end{itemize}
  \item Referencetype transformations:
    \begin{itemize}
      \item \cppid{remove\_cvref<T>}.
    \end{itemize}
  \item Common type transformations:
    \begin{itemize}
      \item \cppid{common\_type<T...>}.
      \item \cppid{common\_reference<T...>}
    \end{itemize}
\end{itemize}
\end{frame}

\begin{frame}[t,fragile]{Deduction transformations: \textbf{void\_t}}
\begin{itemize}
  \item \cppid{void\_t<T...>}: Its internal \cppid{type} is an alias for \cppkey{void}.
    \begin{itemize}
      \item A way to detect a type property.
\begin{lstlisting}
template <typename T, typename = void> 
constexpr bool is_range_v = false;

template <typename T>
constexpr bool is_range_v<T, std::void_t<decltype(std::declval<T>().begin()),
                                         decltype(std::declval<T>().end()>> = true;

void f() {
  static_assert(is_range_v<std::vector<int>>);
  static_assert(not is_range_v<int>); 
}
\end{lstlisting}

      \item \textbad{Note}: Concepts are a much better solution.
    \end{itemize}
\end{itemize}
\end{frame}

\begin{frame}[t,fragile]{Deduction transformations: \textbf{type\_identity\_t}}
\begin{itemize}
  \item \cppid{type\_identity<T>}: Its internal \cppid{type} is an alias for \cppid{T}.
    \begin{itemize}
      \item It can be used to establish a \textmark{non-deduced context} in argument deduction.
\begin{lstlisting}
template <typename T>
void add1(T x, T y) { return x+y; }

template <typename T>
void add2(T x, std::type_identity_t<T> y) { return x+y; }

void f() {
  add1(1.0, 1.5); // OK. T = double
  add1(1.0, 1); // Error T = double / T = int
  add2(1.0, 1); // OK. add2<double>(1.0,1)
}
\end{lstlisting}
    \end{itemize}
\end{itemize}
\end{frame}

\begin{frame}[t,fragile]{Conditional transformations}
\begin{itemize}
  \item \cppid{conditional<B,T1,T2>}
    \begin{itemize}
      \item Its internal alias \cppid{type} depends on the compile time
            boolean \cppid{B}.
        \begin{itemize}
          \item If \cppid{B} is \cppkey{true} $\Rightarrow$ 
                \cppid{type} is \cppid{T1}.
          \item If \cppid{B} is \cppkey{false} $\Rightarrow$ 
                \cppid{type} is \cppid{T2}.
        \end{itemize}

    \end{itemize}

  \mode<presentation>{\vfill\pause}
  \item \cppid{enable\_if<B, T>}
    \begin{itemize}
      \item For backwards compatibility with C++11/C++14/C++17.
      \item Its use cases are better covered with C++20 concepts.
    \end{itemize}
\end{itemize}
\end{frame}
