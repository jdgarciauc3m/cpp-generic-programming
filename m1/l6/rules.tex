\subsection{Rules on NTTP}

\begin{frame}[t,fragile]{Limits on NTTP}
\begin{itemize}
  \item The type for a value of a \textmark{non-type template parameter} can be:
    \begin{itemize}
      \item An integral type.
      \item A floating-point type.
      \item An enumeration type.
      \item A pointer type or pointer to member type.
      \item The \cppid{nullptr\_t} type.
      \item An l-value reference type.
      \item A literal class type.
    \end{itemize}
\end{itemize}
\end{frame}

\begin{frame}[t,fragile]{NTTP of integral types}
\begin{itemize}
  \item An NTTP can be of any integral type.
    \begin{itemize}
      \item \textbad{Note}: Integral types also include \cppkey{bool}.
    \end{itemize}

\begin{columns}[T]

\column{.5\textwidth}
\begin{block}{Controlling output}
\begin{lstlisting}
template <bool formatted, std::integral T>
void print_value(T x) {
  if constexpr (formatted) {
    std::cout << std::format("{:10}\n", x);
  }
  else {
    std::cout << x << '\n';
  }
}
\end{lstlisting}
\end{block}

\column{.5\textwidth}
\begin{block}{Printing values}
\begin{lstlisting}
void f() {
  print_value<true>(42);  // uses format
  print_value<false>(42); // does not use format
  //...
}
\end{lstlisting}
\end{block}

\end{columns}

\end{itemize}
\end{frame}


\begin{frame}[t,fragile]{Taking advantage of default values}

\begin{columns}[T]

\column{.55\textwidth}
\begin{block}{Controlling output}
\begin{lstlisting}
template <std::integral T, 
          bool formatted = (sizeof(T) > sizeof(int)) >
void print_integral(T x) {
  std::cout << std::format("size = {}\n", sizeof(T));
  if constexpr (formatted) {
    std::cout << std::format("{:10}\n", x);
  }
  else {
    std::cout << x << '\n';
  }
}
\end{lstlisting}
\end{block}

\column{.45\textwidth}
\begin{block}{Printing values}
\begin{lstlisting}
void f() {
  print_integral(42); // does not use format
  print_integral(42L); // uses format
  print_integral(42LL); // uses format
  //...
}
\end{lstlisting}
\end{block}

\end{columns}

\end{frame}

\begin{frame}[t,fragile]{NNTP of floating point values}
\begin{itemize}
  \item A NNTP can be of any floating point type.
\end{itemize}

\begin{columns}[T]

\column{.5\textwidth}
\begin{block}{Generic function}
\begin{lstlisting}
template <double value, std::floating_point T>
bool is_greater_fp(T x) {
  return x > value;
}
\end{lstlisting}
\end{block}

\column{.5\textwidth}
\begin{block}{Using a generic function}
\begin{lstlisting}
void f() {
  double x = 5.0;
  std::cout << is_greater_fp<2.0>(x) << '\n';
  //...

  std::vector<double> values{1, 3, 5, 7, 9, 2, 4, 6, 8};
  std::vector<double> results{};
  std::ranges::copy_if(values, 
      std::back_inserter(results), 
      is_greater_fp<4.0, double>);
  //...
}
\end{lstlisting}
\end{block}

\end{columns}
\end{frame}


\begin{frame}[t,fragile]{A logging library with NTTP enums}
\begin{itemize}
  \item Enums can be used at compile time.
\end{itemize}

\mode<presentation>{\vspace{-.5em}}
\begin{columns}[T]

\column{.45\textwidth}
\begin{block}{Converting enum to strings}
\begin{lstlisting}
enum class log_level : std::uint8_t {
  info, warn, error
};

template <log_level level>
consteval std::string_view to_string() {
  switch (level) {
    case log_level::info: return "info";
    case log_level::warn: return "warning!";
    case log_level::error: return "ERROR!!!";
  }
}
\end{lstlisting}
\end{block}

\pause
\column{.55\textwidth}
\begin{block}{Printing log messages}
\begin{lstlisting}[escapechar=@]
template <log_level level>
void log(std::string_view message) {
  std::cout << std::format("{}: {}\n", to_string<level>(), message);
}
@\pause@
void f() {
  using enum log_level;
  log<info>("Starting function");
  log<warn>("Function is deprecated");
  log<error>("Stop using this function");
  log<info>("Finishing function");
}
\end{lstlisting}
\end{block}

\end{columns}

\end{frame}


\begin{frame}[t,fragile]{NTTP of pointer types}
\begin{itemize}
  \item A NTTP may be of any pointer or pointer to member type.
    \begin{itemize}
      \item The pointed object must have external, internal linkage or no linkage.
    \end{itemize}
\end{itemize}

\begin{columns}[T]

\column{.5\textwidth}
\begin{block}{Generic function}
\begin{lstlisting}
struct value_limits {
    double low;
    double high;
};

template <value_limits * limits>
double limit_value(double x) {
  static_assert(limits !=nullptr);
  if (x < limits->low) { return limits->low; }
  if (x > limits->high) { return limits->high; }
  return x;
}
\end{lstlisting}
\end{block}

\pause
\column{.5\textwidth}
\begin{block}{Using a generic function}
\begin{lstlisting}[basicstyle=\tiny]
value_limits lim1{1.0, 5.0};

void f1() {
  std::cout << limit_value<&lim1>(0.0) << '\n';

  static value_limits lim2{3.0, 7.0};
  std::cout << limit_value<&lim2>(10.0) << '\n';

  value_limits lim3{-3.0, 3.0};
  std::cout << limit_value<&lim3>(10.0) << '\n'; // Error

  constexpr value_limits * pnull = nullptr;
  std::cout << limit_value<pnull>(10.0) << '\n'; // Assert
}

\end{lstlisting}
\end{block}

\end{columns}
\end{frame}

\begin{frame}[t,fragile]{NTTP and string literals}
\begin{itemize}
  \item String literals are pointers.
\end{itemize}

\begin{columns}[T]

\column{.5\textwidth}
\begin{block}{Generic function}
\begin{lstlisting}
template <char const * prefix>
void print_message(std::string_view message) {
  std::cout << 
      std::format("{}: {}\n", prefix, message);
}

\end{lstlisting}
\end{block}

\pause
\column{.5\textwidth}
\begin{block}{Using string literals}
\begin{lstlisting}
char const info[] = "INFO"; 

void f1() {
  print_message<info>("Start");

  static char const warn[] = "WARN";
  print_message<warn>("No linkage");

  char const error[] = "ERROR";
  print_message<error>("local"); // Error

  print_message<"ERROR">("\Literal"); // Error
}
\end{lstlisting}
\end{block}

\end{columns}
\end{frame}

\begin{frame}[t,fragile]{NTTP of class types}
\begin{itemize}
  \item A NTTP parameter may be of \textemph{literal class type} where all \textmark{members are public}.
  
  \mode<presentation>{\vfill\pause}
  \item A \textmark{literal class type}:
    \begin{itemize}
      \item Has a \cppkey{constexpr} destructor.
      \item Is one of the following:
        \begin{itemize}
          \item A lambda type.
          \item An aggregate union type with 
                no variant members or one variant member of non-volatile literal type.
          \item A non-union aggregate type where anonymous union members have
                no variant members or one variant member of non-volatile literal type.
          \item A type with at least one \cppkey{constexpr} constructor that is not copy/move.
        \end{itemize}
    \end{itemize}
\end{itemize}
\end{frame}

\begin{frame}[t,fragile]{NTTP and literal class types}
\mode<presentation>{\vspace{-1em}}
\begin{columns}[T]

\column{.5\textwidth}
\begin{block}{Generic function}
\begin{lstlisting}
struct value_limits {
    constexpr value_limits(double vlow, double vhigh)
      : low{vlow}, high{vhigh} 
    {
      Expects(vlow <= vhigh);
    }
    double low;
    double high;
};

template <value_limits limits>
double limit_value(double x) {
  if (x < limits.low) { return limits.low; }
  if (x > limits.high) { return limits.high; }
  return x;
}
\end{lstlisting}
\end{block}

\column{.5\textwidth}
\begin{block}{Using a generic function}
\begin{lstlisting}[basicstyle=\tiny]
void f2() {
  constexpr value_limits lim1{1.0, 3.0};
  
  std::cout << lim1 << limit_value<lim1>(5.0) << '\n';

  std::cout << limit_value<value_limits{3.0, 8.0}>(5.0) << '\n';
}
\end{lstlisting}
\end{block}

\end{columns}
\end{frame}
