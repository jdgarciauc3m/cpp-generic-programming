\subsection{Introduction}

\begin{frame}[t,fragile]{Parameters in a template}
\begin{itemize}
  \item A function template may have mutiple parameters.

\begin{block}{Function with several parameters}
\begin{lstlisting}
template <arithmetic T>
void print_squares_sum(T x, T y) {
  std::cout << std::format("|<{},{}>|^2 = {}\n", x, y, x*x+y*y);
}
\end{lstlisting}
\end{block}

  \item Parameters:
    \begin{itemize}
      \item \textmark{Function template parameter}: \cppid{T}.
      \item \textmark{Function invocation parameters}: \cppid{x} and \cppid{y}.
    \end{itemize}

  \item \textbad{Note}:
    \begin{itemize}
      \item \cppid{x} and \cppid{y} must be of the same type \cppid{T}.
    \end{itemize}

\end{itemize}
\end{frame}

\begin{frame}[t,fragile]{Function template parameters and invocation}
\begin{itemize}
  \item When a function template is instantiated all function invocation parameters
        for the same type must match that type.

\begin{block}{Function invocation}
\begin{lstlisting}
void f() {
  print_squares_sum(2,3); // OK. T = int
  print_squares_sum(2,3.0); // Error T = int / T = double
}
\end{lstlisting}
\end{block}

  \mode<presentation>{\vfill\pause}
  \item \textgood{Solution}:
    \begin{itemize}
      \item Use different function template parameters.
    \end{itemize}
\end{itemize}
\end{frame}
