\subsection{Return type of a generic function}

\begin{frame}[t,fragile]{Returning a value}

\mode<presentation>{\vspace{-1em}}

\begin{itemize}
  \item With single function template parameters there is a single choice.

\begin{block}{Return type with on template parameter}
\begin{lstlisting}
template<arithmetic T>
T modulo(T x, T y) {
  return x * x + y * y;
}
\end{lstlisting}
\end{block}

  \mode<presentation>{\vfill\pause}
  \item Return type will be the same as template function 

\begin{block}{Invocation and return type}
\begin{lstlisting}
void f() {
  auto a = modulo(2,3); // a is int. Value is 13
  auto b = modulo(2.1, 3.2); // b is double. Value is 14.65
  auto c = modulo(2.1, 3); // Error T = double / T = int
}
\end{lstlisting}
\end{block}

\end{itemize}
\end{frame}

\begin{frame}[t,fragile]{Choices for return type with 2 template parameters}

\mode<presentation>{\vspace{-1em}}
\begin{itemize}
  \item If a function has more than one template parameter, a choice is needed.

\begin{block}{A choice for the return type}
\begin{lstlisting}
template<arithmetic T, arithmetic U>
T modulo(T x, U y) {
  return x * x + y * y;
}
\end{lstlisting}
\end{block}

  \mode<presentation>{\vfill\pause}
  \item \textbad{Danger}: Be careful with the result type.

\begin{block}{Invocation and result type}
\begin{lstlisting}
void f() {
  auto a = modulo(2,3); // a is int. Value is 13
  auto b = modulo(2.1, 3); // b is double. Value 13.41
  auto c = modulo(2, 3.1); // c is double. Value is 13. Ouch!!!
}
\end{lstlisting}
\end{block}

\end{itemize}
\end{frame}

\begin{frame}[t,fragile]{Alternatives for return type}
\begin{itemize}
  \item Solutions to deal with return type:
    \begin{itemize}
      \item An additional function template parameter.
\begin{lstlisting}
template <typename R, typename T, typename U>
R func(T x, U y);
\end{lstlisting}

      \mode<presentation>{\vfill\pause}
      \item Find a common type with \cppid{std::common\_type}.
\begin{lstlisting}
template <typename T, typename U>
std::common_type<T,U> func(T x, U y);
\end{lstlisting}

      \mode<presentation>{\vfill\pause}
      \item Use return type deduction.
\begin{lstlisting}
template <typename T, typename U>
auto func(T x, U y);
\end{lstlisting}

    \end{itemize}
\end{itemize}
\end{frame}
