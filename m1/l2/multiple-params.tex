\subsection{Generic functions with multiple generic parameters}

\begin{frame}[t,fragile]{A generic maximum function}
\begin{itemize}
  \item A function may take \textmark{multiple} generic parameters.

\begin{block}{A generic maximum function}
\begin{lstlisting}
arithmetic auto maximum(arithmetic auto x, arithmetic auto y) {
  return (x>y) ? x : y;
}
\end{lstlisting}
\end{block}

  \mode<presentation>{\vfill\pause}
  \item Each generic parameter can be of a \textmark{different} type.

\begin{block}{Invoking maximum()}
\begin{lstlisting}
void f() {
  arithmetic auto r1 = maximum(3, 2L); // returns 2 as long
  arithmetic auto r2 = maximum(1.0, 4); // returns 4.0 as double
}
\end{lstlisting}
\end{block}

\end{itemize}
\end{frame}

\begin{frame}[t,fragile]{Repeating the same type}
\begin{itemize}
  \item What if we want both parameters to be from \textmark{exactly} the same type?
    \begin{itemize}
      \item Use the \textemph{template notation}.
    \end{itemize}

\mode<presentation>{\vfill\pause}

\begin{columns}[T]

\column{.4\textwidth}

\begin{block}{A generic maximum function}
\begin{lstlisting}
template <arithmetic T>
T maximum(T x, T y) {
  return (x>y) ? x : y;
}
\end{lstlisting}
\end{block}

\column{.6\textwidth}

\pause
\begin{block}{A generic maximum function}
\begin{lstlisting}
void f() {
  auto r1 = maximum(2,3); // OK, r1 == 3
  auto r2 = maximum(2, 3L); // Conflict T=int or long?
  auto r3 = maximum(2.0, 3); // Conflict T=double, int?
}
\end{lstlisting}

\end{block}

\end{columns}

  \mode<presentation>{\vfill\pause}
  \item \cppid{T} is a type that satisfies the \cppid{arithmetic} concept.
    \begin{itemize}
      \item Parameters \cppid{x} and \cppid{y} must be of the same type \cppid{T}.
      \item Return type is of the same type \cppid{T}.
    \end{itemize} 

\end{itemize}
\end{frame}

\subsection{Constraining multiple generic parameters}

\begin{frame}[t,fragile]{Extending maximum() for more types}
\begin{itemize}
  \item Currently \cppid{maximum()} only works for arithmetic types.
    \begin{itemize}
      \item What about other comparable types?
        \begin{itemize}
          \item \cppid{std::string}, \cppid{std::tuple}, \ldots
        \end{itemize}
    \end{itemize}

  \mode<presentation>{\vfill\pause}
  \item \textbad{Idea}: Use \cppid{std::totally\_ordered\_with<T,U>} concept.
    \begin{itemize}
      \item Types \cppid{T} and \cppid{U} satisfy the \cppid{std::totally\_ordered\_with} concept
            if:
        \begin{itemize}
          \item All comparison operators for a value \cppid{x} of type \cppid{T} and
                a value \cppid{y} of type \cppid{U} results are consistent 
                with a \textmark{strict total order}.
          \item Comparison operators are:
                \cppkey{==}, \cppkey{!=}, \cppkey{<}, \cppkey{<=}, \cppkey{>}, and \cppkey{>=}.
        \end{itemize}
    \end{itemize}

  \mode<presentation>{\vfill\pause}
  \item What about the result type?
    \begin{itemize}
      \item Use \cppid{common\_type\_t<T,U>}.
        \begin{itemize}
          \item A type \cppid{R} that types \cppid{T} and \cppid{U} can be implicitly converted to.
        \end{itemize}
    \end{itemize}

\end{itemize}
\end{frame}

\begin{frame}[t,fragile]{Extended template notation}
\begin{block}{An extended generic function}
\begin{lstlisting}
template <typename T, typename U>
  requires totally_ordered_with<T,U>
std::common_type_t<T,U> maximum(T x, U y) {
  return (x>y) ? x : y;
}
\end{lstlisting}
\end{block}

\begin{itemize}
  \mode<presentation>{\vfill\pause}
  \item Keyword \cppkey{typename} means that the \textmark{template parameter} can be
        \textgood{any type}.

  \mode<presentation>{\vfill}
  \item The \cppkey{requires} clause is useful for expressing requirements on multiple
        generic parameters.
\end{itemize}

\end{frame}

\begin{frame}[t,fragile]{Adding constraints}

\begin{block}{Maximum on strings versus pointers}
\begin{lstlisting}
void f() {
  using namespace std::literals;
  auto r1 = maximum("carlos"s,"daniel"s); // r1 = "daniel"

  auto r2 = maximum("carlos","daniel"); // r2 = "carlos"
}  
\end{lstlisting}
\end{block}

\begin{itemize}
  \item When \cppid{maximum} is invoked with two \cppid{std::string} objects,
        operator \cppkey{<} is invoked and lexicographical order is applied.

  \mode<presentation>{\vfill}
  \item When \cppid{maximum} is invoked with two \cppid{char*} objectis,
        operator \cppkey{<} is invoked and memory addresses order is applied.
\end{itemize}

\end{frame}


\begin{frame}[t,fragile]{Constraining types}
\begin{block}{Constraining with static assertions}
\begin{lstlisting}
template<typename T, typename U>
    requires std::totally_ordered_with<T, U>
std::common_type_t<T, U> maximum(T x, U y) {
  static_assert(not std::is_pointer_v<T>, "T must not be a pointer type");
  static_assert(not std::is_pointer_v<U>, "T must not be a pointer type");
  return (x > y) ? x : y;
}
\end{lstlisting}
\end{block}

\begin{itemize}
  \item The following constraint must be satisfied:
    \begin{itemize}
      \item Types \cppid{T} and \cppid{U} must define a total order.
    \end{itemize}
  \item The following additional compile-time checks are performed.
    \begin{itemize}
      \item Type \cppid{T} must not be a pointer type.
      \item Type \cppid{U} must not be a pointer type.
    \end{itemize}
\end{itemize}
\end{frame}

\begin{frame}[t,fragile]{Constraining types}
\begin{block}{A more constrained generic function}
\begin{lstlisting}
template<typename T, typename U>
    requires std::totally_ordered_with<T, U>
        and (not std::is_pointer_v<T>)
        and (not std::is_pointer_v<U>)
std::common_type_t<T, U> maximum(T x, U y) {
  return (x > y) ? x : y;
}
\end{lstlisting}
\end{block}

\begin{itemize}
  \item Now the following constraints must be satisfied:
    \begin{itemize}
      \item Types \cppid{T} and \cppid{U} must define a total order.
      \item Types \cppid{T} must not be a pointer type.
      \item Types \cppid{U} must not be a pointer type.
    \end{itemize}
\end{itemize}
\end{frame}
