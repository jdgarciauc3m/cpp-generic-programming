\subsection{Additional template parameter for return type}

\begin{frame}[t,fragile]{Explicitly specifying template arguments}
\begin{itemize}
  \item When a generic function is invoked its template arguments are deduce.
    \begin{itemize}
      \item However they can be explicitly specified.
    \end{itemize}

\begin{block}{Explict function call}
\begin{lstlisting}
template <arithmetic T>
void print_modulo(T x, T y) {
  std::cout << "| <" << x << "," << y << "> | = " << x*x + y*y <<'\n';
}

void f() {
  print_modulo(2,3); // T = int
  print_modulo(2.0,3);  // Error. T = double / T = int
  print_modulo<double>(2.0,3); // OK. T = double. 3 -> double
  print_modulo<int>(2.0,3); // OK. T = int. 2.0 -> int
}

\end{lstlisting}
\end{block}

\end{itemize}
\end{frame}

\begin{frame}[t,fragile]{Explicitly specifying multiple parameters}
\begin{itemize}
  \item Multiple template arguments can be explicitly specified.

\begin{block}{Explicit function call and multiple parameters}
\begin{lstlisting}
template <arithmetic T, arithmetic U>
void print_modulo(T x, U y) {
  std::cout << "| <" << x << "," << y << "> | = " << x*x + y*y <<'\n';
}

void f() {
  print_modulo(2,3.1); // T = int, U = double
  print_module<int>(2, 3.1); // T = int, U = double
  print_module<int,double>(2, 3.1); // T = int, U = double
  print_module<int,int>(2, 3.1); // T = int, U = int
}
\end{lstlisting}
\end{block}

\end{itemize}
\end{frame}

\begin{frame}[t,fragile]{Providing a function template parameter for return type}

\mode<presentation>{\vspace{-1em}}
\begin{itemize}
  \item An extra function template parameter may be used for the return type.

\begin{block}{A return type template parameter}
\begin{lstlisting}
template<arithmetic T, arithmetic U, arithmetic R>
R modulo(T x, U y) {
  return x * x + y * y;
}
\end{lstlisting}
\end{block}

  \mode<presentation>{\vfill\pause}
  \item Function template argument deduction does not apply to return type.

\begin{block}{Invocation}
\begin{lstlisting}
void f() {
  auto a = modulo<double, int, double>(2.0, 3); // OK. R = double
  auto b = modulo<double, int>(2.0, 3); // Error: Cannot deduce R
  auto c = modulo(2.0, 3); // Cannot deduce T, U, R
}
\end{lstlisting}
\end{block}

\end{itemize}
\end{frame}

\begin{frame}[t,fragile]{Making argument deduction simpler}
\begin{itemize}
  \item \textgood{Idea}: Make the return type the first parameter.
    \begin{itemize}
      \item We can still deduce other arguments.
    \end{itemize}
\begin{block}{A return type template parameter}
\begin{lstlisting}
template<arithmetic R, arithmetic T, arithmetic U>
R modulo(T x, U y) {
  return x * x + y * y;
}

void f() {
  auto a = modulo<double, int, double>(2.0, 3); // OK. R = double
  auto b = modulo<double>(2.0, 3); // OK. R= double, T = double, U = int
  auto c = modulo(2.0, 3); // Cannot deduce R, T, U
}
\end{lstlisting}
\end{block}

\end{itemize}
\end{frame}
