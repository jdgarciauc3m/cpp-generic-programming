\subsection{Type inference from initializer}

\begin{frame}[t,fragile]{What is the type of that variable?}
  \begin{itemize}
    \item When a variable is defined and initialized, 
          the compiler must check that the type of the variable
          matches the type of the initialization expression.
      \begin{itemize}
        \item Compiler already knows the variable's type.
        \item The variable's type can be deduced at compile time.
      \end{itemize}
  \end{itemize}
\pause
\begin{lstlisting}[escapechar=@]
double x = 1.5;@\pause@
auto x = 1.5; // Type of x is deduced to double
\end{lstlisting}
\pause
  \begin{itemize}
    \item Type inference may avoid code that is specially 
       complex or difficult to maintain.
  \end{itemize}
\pause
\begin{lstlisting}[escapechar=@]
typename std::vector<std::string>::iterator i= c.begin();@\pause@
typename std::vector<std::string>::const_iterator i=c.begin();@\pause@
auto i = c.begin();
\end{lstlisting}
\end{frame}

\begin{frame}[t,fragile]{\texttt{auto} definitions}
\begin{itemize}
\item Specially useful in generic code.
\begin{lstlisting}[escapechar=@]
template <typename T>
void print1(const std::vector<T> & v) {
  for (@\color{red}typename std::vector<T>::const\_iterator@ i=v.begin();i!=v.end();++i){
    std::cout << *i << "\n";
  }
}
\end{lstlisting}
\mode<presentation>{\vfill\pause}
\begin{lstlisting}[escapechar=@]
template <typename T>
void print2(const std::vector<T> & v) {
  for (@\color{red}auto@ i=v.begin();i!=v.end();++i){
    std::cout << *i << "\n";
  }
}
\end{lstlisting}
\end{itemize}
\end{frame}

\begin{frame}[t,fragile]{Example}
  \begin{itemize}
    \item Given two list of numbers with the same length, 
          print corresponding pairs whose addition is positive.
  \end{itemize}
\begin{block}{Homogeneous solution}
\begin{lstlisting}
template <std::integral T>
void print_add(std::vector<T> const & vec1, std::vector<T> const & vec2) {
  for (auto i = vec1.begin(), j = vec2.begin(); i != vec1.end(); ++i, ++j) {
    T sum = *i + *j;
    if (sum > 0) { std::cout << std::format("{} + {} = {}\n", *i, *j, sum); }
  }
}
\end{lstlisting}
\end{block}

\mode<presentation>{\vspace{-1em}}
\begin{columns}[T]
\column{.5\textwidth}
\begin{lstlisting}[basicstyle=\tiny]
void test1() {
  std::vector<int> v1{1, -1, 2, - 2};
  std::vector<int> v2{1, 1, 1, 1};
  print_add(v1,v2);
\end{lstlisting}

\column{.5\textwidth}
\begin{lstlisting}[basicstyle=\tiny]
  std::vector<long> v3{1, 1, 1, 1};
  print_add(v1,v3); // Error: Conflicting deduction
}
\end{lstlisting}
\end{columns}

\end{frame}

\begin{frame}[t,fragile]
\begin{block}{Heterogeneous solution}
\begin{lstlisting}[escapechar=@]
template <std::integral T, std::integral U>
void print_add(std::vector<T> const & vec1, std::vector<U> const & vec2) {
  auto i = vec1.begin();
  auto j = vec2.begin();
  for (; i != vec1.end(); ++i, ++j) {
    @\color{red}auto@ sum = *i + *j;
    if (sum > 0) { std::cout << std::format("{} + {} = {}\n", *i, *j, sum); }
  }
}
\end{lstlisting}
\end{block}
\begin{itemize}
  \item An \cppkey{auto} definition may include a concept constraint.
\end{itemize}
\end{frame}

\begin{frame}[t,fragile]{Deduction}
\mode<presentation>{\vspace{-1em}}
  \begin{itemize}
    \item \cppkey{auto} admits modifiers:
      \begin{itemize}
        \item \cppkey{const} and \cppkey{volatile}.
        \item \emph{reference} and \emph{pointer}.
        \item Same rules as template argument deduction.
      \end{itemize}
  \end{itemize}
\begin{lstlisting}[escapechar=@]
int f();
auto x1 = f(); // int x1
const auto & x2 = f(); // const int & x2. Lifetime extension.
auto & x3 = f(); // Error. Cannot bind int& to temporal.
@\pause@
double & g();
auto y1 = g(); // double y1
const auto & y2 = g(); // const double & y2
auto & y3 = g(); // double & y3
auto * y4 = g(); // Error g() does not return a pointer
@\pause@
std::string * h();
auto s1 = h(); // string * s1
auto * s2 = h(); // string * s2
\end{lstlisting}
\end{frame}

\begin{frame}[fragile]{\texttt{auto} deduction and references}
  \begin{itemize}
    \item Value semantics is default option.
    \item Reference semantics is obtained by explicitly specifying \cppkey{auto\&}.
  \end{itemize}

\begin{columns}[T]
\column{.6\textwidth}
\begin{lstlisting}
X f();

X x1 = f(); 
auto x1 = f(); // X x1

X & x2 = f(); // Error. Cannot bind reference to temporal.
auto & x2 = f(); // Error. Cannot bind reference to temporal.

X && x3 = f();
auto && x3 = f(); // X && x1
\end{lstlisting}

\pause
\column{.45\textwidth}
\begin{lstlisting}
X & g();

X x1 = g();
auto x1 = g(); // X x1

X & x2 = g();
auto & x2 = g(); // X & x2

X && x3 = g();
auto && x3 = g(); // X & x3. 
                  // Note X & && -> X &
\end{lstlisting}
\end{columns}
\end{frame}

\begin{frame}[t,fragile]{Range for-loop and auto}
\begin{itemize}
  \item \textmark{Alternatives}:

\begin{lstlisting}[escapechar=@]
for (auto x: expr) { /*...*/} // Work with copies
@\pause@
for (auto const x : expr) { /*...*/} // Useless. Prefer auto const &
@\pause@
for (auto & x: expr) { /*...*/} // Modify elements in non-generic code
@\pause@
for (auto const & x : expr) { /*...*/} // Read-only access generic/non-generic code (includes proxies)
@\pause@
for (auto && x : expr) { /*...*/ } // Modify elements in generic code
@\pause@
for (auto const && x : expr) { /*...*/ } // Useles. Prefer auto const &
\end{lstlisting}

\end{itemize}
\end{frame}

\begin{frame}[fragile]{\texttt{auto} and multiple declarations}
\begin{itemize}
  \item Multiple variables can be declared in a single declaration statement with \cppkey{auto}.
  \item All deductions in the same statement must result in the same type.
\end{itemize}
\begin{lstlisting}
auto x = f(2.5), y = f(3.0);
auto s = 1.5, t = 1.5f; // Error
auto z1= 10, z2; // Error z2 without initializer
\end{lstlisting}
\begin{itemize}

  \mode<presentation>{\vfill\pause}
  \item Even if it does not imply that all of then must have the same type.
\end{itemize}
\begin{lstlisting}
int v[] = {1, 2, 3};
auto x = v[0], *p = &v[0];
\end{lstlisting}
\begin{itemize}
  \item Declarations are handled left to right.
\end{itemize}
\begin{lstlisting}
auto x=1, *p = &x;
\end{lstlisting}
\end{frame}

