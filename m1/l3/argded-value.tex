\subsection{Argument deduction and by-value parameters}

\begin{frame}[t,fragile]{Argument deduction rules}
\begin{itemize}
  \item When a generic function is invoked template parameters are
        \textmark{deduced} from the invocation arguments.

\begin{columns}[T]

\column{.5\textwidth}
\begin{block}{A generic maximum}
\begin{lstlisting}
template <typename T>
T maximum(T x, T y) {
  return x>y ? x : y;
}
\end{lstlisting}
\end{block}

\column{.5\textwidth}
\begin{block}{Type deduction}
\begin{lstlisting}
void f() {
  int a = 2, b = 5;
  auto r1 = maximum(a,b); // T = int
}
\end{lstlisting}
\end{block}
\end{columns}

  \mode<presentation>{\vfill\pause}
  \item \textmark{Decay conversions} are supported.
    \begin{itemize}
      \item Conversions applied in function argument passing.
    \end{itemize}

\end{itemize}
\end{frame}

\begin{frame}[t,fragile,allowframebreaks]{Deducing template parameters}

\begin{itemize}
  \item References to \cppid{T} \textmark{decay} to \cppid{T}
    \begin{itemize}
      \item Applies to l-value and r-value references.
    \end{itemize}
\end{itemize}

\begin{block}{Deducing references}
\begin{lstlisting}
void f() {
  int x = 3, y = 5;
  int & rx = x;
  int && rrz = x + y;

  auto r1 = maximum(x, rx); // T = int
  auto r2 = maximum(rrz, y); // T = int
}
\end{lstlisting}
\end{block}

\framebreak

\begin{itemize}
  \item A type \cppid{T} which is \cppkey{const} and/or \cppkey{volatile}
        \textmark{decays} in the corresponding type without such qualification.
\end{itemize}

\begin{block}{Deducing with \textbf{const}/\textbf{volatile}}
\begin{lstlisting}
void f() {
  int x = 3, y = 5;
  int const cx = x;
  int const & crx = x;
  int const && rrz = x + y;
  auto r1 = maximum(cx, y); // T = int
  auto r2 = maximum(crx, y); // T = int
  auto r3 = maximum(rrz, y); // T = int
}
\end{lstlisting}
\end{block}

\framebreak

\begin{itemize}
  \item An array of \cppid{T} \textmark{decays} to \cppid{T*}.
    \begin{itemize}
      \item You can pass arrays of different sizes.
      \item Array size is \textbad{lost}.
    \end{itemize}
\end{itemize}

\begin{block}{Deducing an array}
\begin{lstlisting}
void f() {
  int v[10];
  int w[5];
  auto r1 = maximum(v,w); // T = int*
}
\end{lstlisting}
\end{block}

\framebreak

\begin{itemize}
  \item A reference to array of \cppid{T} \textmark{decays} to \cppid{T*}.
\end{itemize}

\begin{block}{Deducing a reference to an array of \textbf{T}}
\begin{lstlisting}
void f() {
  int v[10];
  auto & s = v; // int (&s)[10]
  auto r1 = maximum(v,s); // T = int*
}
\end{lstlisting}
\end{block}

\framebreak

\begin{itemize}
  \item But an array of \cppid{T const} \textmark{decays} to \cppid{T const *}.
    \begin{itemize}
      \item \cppid{T *} and \cppid{T const *} are different types.
    \end{itemize}
\end{itemize}

\begin{block}{Deducing an array of \textbf{T const}}
\begin{lstlisting}
void f() {
  int v[10];
  int const w[10];
  auto r1 = maximum(v,s); // Error: T=int* / T=int const *
}
\end{lstlisting}
\end{block}

\framebreak

\begin{itemize}
  \item However, a \cppid{T * const} \textmark{decays} to \cppid{T *}.
    \begin{itemize}
      \item Top level \cppkey{const} and \cppkey{volatile} are removed during \textmark{decay}.
    \end{itemize}
\end{itemize}

\begin{block}{Deducing a \textbf{const} pointer to \textbf{T}}
\begin{lstlisting}
void f() {
  int v[10];
  int * const p = v;
  auto r1 = maximum(v,p); // T=int*
}
\end{lstlisting}
\end{block}

\framebreak

\begin{itemize}
  \item A \textemph{function} \textmark{decays} into a \textemph{function pointer}.

  \item A \textemph{function reference} \textmark{decays} into a \textemph{function pointer}. 
\end{itemize}

\begin{block}{Deducing functions and reference to functions}
\begin{lstlisting}
long add(long x, long y) { return x+y; }
long sub(long x, long y) { return x-y; }

void f() {
  auto r = maximum(add, sub); // T=long (*)(long,long)

  auto * fun1 = add; // long (*)(long long)
  auto & fun2 = sub; // long (&)(long long)
  auto s = maximum(add,sub); // T = long (*)(long,long)
}
\end{lstlisting}
\end{block}

\end{frame}
