\subsection{Combining concepts}

\begin{frame}[t,fragile]{A missing concept}
\begin{itemize}
  \item The standard library does not have arithmetic concept.
    \begin{itemize}
      \item \cppid{std::integral}: A concept for integral types.
      \item \cppid{std::floating\_point}: A concept for floating point types.
    \end{itemize}

  \mode<presentation>{\vfill\pause}
  \item \textbad{Problem}: Need to define a generic function twice.
\end{itemize}

\mode<presentation>{\vfill\pause}

\begin{block}{Square function duplication}
\begin{lstlisting}
std::integral auto square(std::integral auto x) {
  return x*x;
}

std::floating_point auto square(std::floating_point auto x) {
  return x*x;
}
\end{lstlisting}
\end{block}
\end{frame}

\begin{frame}[t,fragile]{A new concept}
\begin{itemize}
  \item A \cppkey{concept} can be defined as a predicate on a type.
    \begin{itemize}
      \item The predicate is evaluated at compile time.
      \item If it is true the type satisfies the concept.
    \end{itemize}

  \mode<presentation>{\vfill\pause}
  \item We can take advantage of the \cppid{std::is\_arithmetic\_v<T>} type trait.
\end{itemize}

\mode<presentation>{\vfill\pause}
\begin{columns}[T]

\column{.5\textwidth}
\begin{block}{Concept definition}
\begin{lstlisting}
template <typename T>
concept arithmetic = std::is_arithmetic_v<T>;
\end{lstlisting}
\end{block}

\pause
\column{.5\textwidth}
\begin{block}{Square function}
\begin{lstlisting}
arithmetic auto square(arithmetic auto x) {
  return x*x;
}
\end{lstlisting}
\end{block}

\end{columns}
\end{frame}
