\subsection{Combining concepts}

\begin{frame}[t,fragile]{A missing concept}
\begin{itemize}
  \item The standard library does not have arithmetic concept.
    \begin{itemize}
      \item \cppid{std::integral}: A concept for integral types.
      \item \cppid{std::floating\_point}: A concept for floating point types.
    \end{itemize}

  \mode<presentation>{\vfill\pause}
  \item \textbad{Problem}: Need to define a generic function twice.
\end{itemize}

\mode<presentation>{\vfill\pause}

\begin{block}{Square function duplication}
\begin{lstlisting}
void print_square(std::integral auto x) {
  std::cout << std::format("{}^2 = {}\n", x, x*x);
}

void print_square(std::floating_point auto x) {
  std::cout << std::format("{}^2 = {}\n", x, x*x);
}
\end{lstlisting}
\end{block}
\end{frame}

\begin{frame}[t,fragile]{Concept combination}
\begin{itemize}
  \item A \cppkey{concept} can be defined as a logical combination of other concepts.
    \begin{itemize}
      \item Logical connectives \cppkey{and}, \cppkey{or}, and \cppkey{not} can be used.
    \end{itemize}
\end{itemize}

\mode<presentation>{\vfill\pause}
\begin{columns}[T]

\column{.5\textwidth}
\begin{block}{Concept definition}
\begin{lstlisting}
template <typename T>
concept arithmetic = 
    std::integral<T> or 
    std::floating_point<T>;
\end{lstlisting}
\end{block}

\pause
\column{.5\textwidth}
\begin{block}{Square function}
\begin{lstlisting}
void print_square(arithmetic auto x) {
  std::cout << std::format("{}^2 = {}\n", x, x*x);
}
\end{lstlisting}
\end{block}

\end{columns}
\end{frame}

\begin{frame}[t,fragile]{A concept from a type trait}
\begin{itemize}
  \item A \cppkey{concept} can be defined as a predicate on a type.
    \begin{itemize}
      \item The \textmark{predicate} is \textgood{evaluated at compile time}.
      \item If it is \cppkey{true} the type \textgood{satisfies the concept}.
    \end{itemize}

  \mode<presentation>{\vfill\pause}
  \item We can take advantage of the \cppid{std::is\_arithmetic\_v<T>} type trait.
    \begin{itemize}
      \item Defined in \cppid{<type\_traits>} header.
    \end{itemize}
\end{itemize}

\mode<presentation>{\vfill\pause}
\begin{columns}[T]

\column{.5\textwidth}
\begin{block}{Concept definition}
\begin{lstlisting}
template <typename T>
concept arithmetic = std::is_arithmetic_v<T>;
\end{lstlisting}
\end{block}

\pause
\column{.5\textwidth}
\begin{block}{Square function}
\begin{lstlisting}
void print_square(arithmetic auto x) {
  std::cout << std::format("{}^2 = {}\n", x, x*x);
}
\end{lstlisting}
\end{block}

\end{columns}
\end{frame}
