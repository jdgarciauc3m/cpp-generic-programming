\subsection{Introduction}

\begin{frame}[t,fragile]{Overload resolution and generic programming}
\begin{itemize}
  \item A generic function may have a non-generic overload.

\begin{block}{Overloading a generic function}
\begin{lstlisting}
double add_squares(double x, double y) {
  return x * x + y * y;
}

template <arithmetic T>
T add_squares(T x, T y) {
  return x * x + y * y;
}
\end{lstlisting}
\end{block}

  \mode<presentation>{\vfill\pause}
  \item \textbad{But}:
    \begin{itemize}
      \item How does selection work?
    \end{itemize}

\end{itemize}
\end{frame}

\begin{frame}[t,fragile]{Better match}
\begin{itemize}
  \item A non-generic version is always preferred to a generic version.
    \begin{itemize}
      \item But if the generic leads to a better match, it is preferred.
    \end{itemize}

\begin{block}{Non-generic better match}
\begin{lstlisting}
void f() {
  auto r1 = add_squares(2.0, 3.5); // add_squares(double,double)

  auto r2 = add_squares(2, 3); // add_squares<int>(int,int)
  auto r3 = add_squares(2.5f, 3.5f); // add_squares<float>(float,float)

  auto r4 = add_squares(2, 3.5); // add_squares(double,double)
}
\end{lstlisting}
\end{block}
\end{itemize}
\end{frame}

\begin{frame}[t,fragile]{Only generic better match}
\begin{itemize}
  \item If an empty angle-bracket is used, only generic versions are considered:

\begin{block}{Generic better match}
\begin{lstlisting}
void f() {
  auto r1 = add_squares<>(2.0, 3.5); // add_squares<double>(double,double)

  auto r2 = add_squares<>(2, 3); // add_squares<int>(int,int)
}
\end{lstlisting}
\end{block}

  \mode<presentation>{\vfill\pause}
  \item However, type conversions are not considered in templates.

\begin{block}{Generic better match}
\begin{lstlisting}
void f() {
  auto r1 = add_squares<>(2, 3.5); // Error: T=int / T=double
}
\end{lstlisting}
\end{block}

\end{itemize}
\end{frame}
