\subsection{Generic overloads and pointers}

\begin{frame}[t,fragile]{Printing values}
\begin{itemize}
  \item A generic overload can be used to print values and pointers.

\begin{block}{Printing values and pointers}
\begin{lstlisting}
void print(auto x) {
  std::cout << "value = " << x << '\n';
}

void print(auto * p) {
  std::cout << "address: " << static_cast<void*>(p);
  std::cout << ". value = " << *p << '\n';
}
\end{lstlisting}
\end{block}

\end{itemize}
\end{frame}

\begin{frame}[t,fragile]{Using a generic printer}
\begin{block}{Invocations for a generic printer}
\begin{lstlisting}
void f() {
  int x = 42:
  print(x); // print<int>(int)

  int * px = &x;
  print(px); // print<int>(int*)

  using namespace std::literals
  print("Daniel"s); // print<std::string>(std::string)

  print("Daniel"); // Error. Cannot cast char const * to void *
}
\end{lstlisting}
\end{block}
\end{frame}

\begin{frame}[t,fragile]{Overload for C-strings}
\begin{itemize}
  \item A non-generic overload added for C-strings.

\begin{block}{Printing values and pointers}
\begin{lstlisting}
void print(auto x) {
  std::cout << "value = " << x << '\n';
}

void print(auto * p) {
  std::cout << "address: " << static_cast<void*>(p);
  std::cout << ". value = " << *p << '\n';
}

void print(char const * s) {
  std::cout << "string = \"" << s << "\"\n";
}
\end{lstlisting}
\end{block}
\end{itemize}
\end{frame}

\begin{frame}[t,fragile]{Using a generic printer}
\begin{block}{Invocations for a generic printer}
\begin{lstlisting}
void f() {
  int x = 42:
  print(x); // print<int>(int)

  int * px = &x;
  print(px); // print<int>(int*)

  using namespace std::literals
  print("Daniel"s); // print<std::string>(std::string)

  print("Daniel"); // print(char const *)
}
\end{lstlisting}
\end{block}
\end{frame}

\begin{frame}[t,fragile,shrink=15]{Print composition}
\begin{itemize}
  \item A multi-value print can be defined in terms of simpler functions.

\begin{columns}[T]

\column{.5\textwidth}
\begin{block}{A multi-value print}
\begin{lstlisting}
void print(auto x) {
  std::cout << "value = " << x << '\n';
}

void print(auto * p) {
  std::cout << "address: " << static_cast<void const*>(p);
  std::cout << ". value = " << *p << '\n';
}

void print(auto x, auto y) {
  print(x);
  print(y);
}

void f() {
  print(42, "Daniel");
}
\end{lstlisting}
\end{block}

\column{.5\textwidth}

\begin{block}{Output}
\begin{lstlisting}[style=terminal]
value = 42
address = 0x618b9af16008. value = D
\end{lstlisting}
\end{block}

\end{columns}

\end{itemize}
\end{frame}

\begin{frame}[t,fragile,shrink=15]{A multi-value print and order}
\begin{itemize}

  \item Order of definition is important.

\begin{columns}[T]

\column{.5\textwidth}
\begin{block}{A multi-value print}
\begin{lstlisting}
void print(auto x) {
  std::cout << "value = " << x << '\n';
}

void print(auto x, auto y) {
  print(x);
  print(y);
}

void print(auto * p) {
  std::cout << "value = " << *p << '\n';
}

void f() {
  print(42, "Daniel");
}
\end{lstlisting}
\end{block}

\column{.5\textwidth}
\begin{block}{Output}
\begin{lstlisting}[style=terminal]
value = 42
value = Daniel
\end{lstlisting}
\end{block}

\end{columns}
\end{itemize}
\end{frame}
