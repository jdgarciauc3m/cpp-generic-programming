\subsection{Overloading templates and temporaries}

\begin{frame}[t,fragile]{Overloading by value}

\begin{columns}[T]

\column{.55\textwidth}
\begin{block}{Overloads of \textbf{maximum}}
\begin{lstlisting}
template <std::totally_ordered T>
T maximum(T x, T y) {
  return x > y ? x : y;
}

template <typename T>
T * maximum(T * px, T * py) {
  return (*px > * py) ? px : py;
}

char const * maximum(char const * s1, char const * s2) {
  return std::strcmp(s1,s2)>0 ? s1 : s2;
}
\end{lstlisting}
\end{block}

\pause
\column{.45\textwidth}
\begin{block}{Invoking maximum}
\begin{lstlisting}
void f() {
  auto r1 = maximum(2,4); // max of ints

  double a = 2.5, b = 3.2;
  double *pa = &a, *pb = &b;
  auto r2 = maximum(pa,pb); // max of pointers

  // Max of C-strings
  auto r3 = maximum("Daniel", "Carlos");
}
\end{lstlisting}
\end{block}

\end{columns}
\end{frame}

\begin{frame}[t,fragile]{A multi-value maximum}
\begin{itemize}
  \item A multi-value \cppid{maximum} defined in terms of simpler versions.

\begin{block}{A maximum of three values}
\begin{lstlisting}
template <typename T>
T maximum(T x, T y, T z) {
  return maximum(maximum(x,y),z);
}
\end{lstlisting}
\end{block}

  \item No problem for C-strings.

\begin{block}{Invoking maximum}
\begin{lstlisting}
void f() {
  auto r = maximum("Daniel", "Carlos", "Maria"); // OK
}
\end{lstlisting}
\end{block}

\end{itemize}
\end{frame}

\begin{frame}[t,fragile]{Overloading by reference}

\begin{columns}[T]

\column{.55\textwidth}
\begin{block}{Overloads of \textbf{maximum}}
\begin{lstlisting}
template <std::totally_ordered T>
T const & maximum(T const & x, T const & y) {
  return x > y ? x : y;
}

template <typename T>
T const * maximum(T const * px, T const * py) {
  return (*px > * py) ? px : py;
}

char const * maximum(char const * s1, char const * s2) {
  return (std::strcmp(s1,s2)>0 ? s1 : s2;
}
\end{lstlisting}
\end{block}

\pause
\column{.45\textwidth}
\begin{block}{Invoking maximum}
\begin{lstlisting}
void f() {
  auto r1 = maximum(2,4); // max of ints

  double a = 2.5, b = 3.2;
  double *pa = &a, *pb = &b;
  auto r2 = maximum(pa,pb); // max of pointers

  // Max of C-strings
  auto r3 = maximum("Daniel", "Carlos");
}
\end{lstlisting}
\end{block}

\end{columns}
\end{frame}

\begin{frame}[t,fragile]{A multi-reference maximum}
\mode<presentation>{\vspace{-1em}}
\begin{itemize}
  \item A multi-reference \cppid{maximum} defined in terms of simpler versions.

\begin{block}{A maximum of three values}
\begin{lstlisting}
template <typename T>
T const & maximum(T const & x, T const & y, T const & z) {
  return maximum(maximum(x,y),z);
}
\end{lstlisting}
\end{block}

  \item A \textbad{problem} for C-strings.
    \begin{itemize}
      \item Passes a a reference to a temporary.
    \end{itemize}

\begin{block}{Invoking maximum}
\begin{lstlisting}
void f() {
  char const * s1 = "Daniel", * s2 = "Carlos", * s3 = "Maria";
  auto r = maximum(s1, s2, s3); // Undefined behavior
}
\end{lstlisting}
\end{block}

\end{itemize}
\end{frame}


