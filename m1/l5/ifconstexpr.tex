\subsection{Compile time selection}

\begin{frame}[t,fragile]{Selecting exectuion paths}
\begin{itemize}
  \item When an \cppkey{if} statement selects on a type property 
        condition both branches must be valid.
    \begin{itemize}
      \item That makes conditional code difficult to write.
      \item Might encourage defining \textmark{micro-concepts} for everything.
      \item \textbad{Problem}: All branches need to compile for every instantiation.
    \end{itemize}

\begin{block}{A print function}
\begin{lstlisting}[basicstyle=\tiny]
template <typename T>
void print(T const & value) {
  if  (std::is_array_v<T>) {
    for (auto x : value) { // Error if instantiated with non-range
      std::cout << x << ' ';
    }
  }
  else {
    std::cout << value; // Decays to pointer inf instantiated with arrays
  }
  std::cout << '\n';
}
\end{lstlisting}
\end{block}

\end{itemize}
\end{frame}

\begin{frame}[t,fragile]{A compile-time if statement}
\begin{itemize}
  \item A compile time if is achieved by using \cppkey{constexpr} before the condition
        of an \cppkey{if} statement.
    \begin{itemize}
      \item During instantiation only the selected branch is intantiated.
    \end{itemize}

\begin{block}{A print function}
\begin{lstlisting}[basicstyle=\tiny]
template <typename T>
void print(T const & value) {
  if  constexpr (std::is_array_v<T>) {
    for (auto x : value) { // OK. Only for arrays
      std::cout << x << ' ';
    }
  }
  else {
    std::cout << value; // OK. Only for non-arrays
  }
  std::cout << '\n';
}
\end{lstlisting}
\end{block}

\end{itemize}
\end{frame}
