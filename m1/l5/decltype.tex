\subsection{Declaration types}

\begin{frame}[t,fragile]{Type of a declaration}
  \begin{itemize}
    \item Gets the type of a declaration.
    \item Useful complement to \cppkey{auto}.
    \item Specially useful in \textmark{generic programming}.
  \end{itemize}

\mode<presentation>{\vfill\pause}
\begin{lstlisting}[escapechar=@]
T x; // x is of type T
auto t = x; // t is of type T

U y; // y is of type U
auto z = y; // y is of type U
@\pause@
auto a = x+y; // a is of the same type than x+y; @\pause@
decltype(x+y) b; // b is of the same type than x+y, 
                 // but I do not know its initial value

// Update b
b = x+y;
\end{lstlisting}
\end{frame}

\begin{frame}[t,fragile]{Example: Adding vectors}
\mode<presentation>{\vspace{-1em}}
\begin{block}{Simple case: Homogeneous}
\begin{lstlisting}
template <class T>
std::vector<T> add(const std::vector<T> & v1, const std::vector<T> & v2) 
{
  //...
}
\end{lstlisting}
\end{block}

\mode<presentation>{\vfill\pause}
\begin{block}{Heterogeneous case}
\begin{lstlisting}
template <class T, class U>
std::vector<decltype(T{}+U{})> add(const std::vector<T> & v1, const std::vector<U> & v2) 
{
  //...
}
\end{lstlisting}
\end{block}
\end{frame}

\begin{frame}[t,fragile]{Moving towards suffix notation}
  \begin{itemize}
    \item Using \textgood{trailing return type} allows to write the \textgood{return type}
          \textmark{after} the arguments have been seen.
      \begin{itemize}
        \item \textgood{Important}: 
              The \textgood{trailing return type} is in the \textmark{scope}
              of the function (and can see the arguments).
      \end{itemize}
  \end{itemize}
\pause
\begin{lstlisting}
template <class T, class U>
auto add(const std::vector<T> & v1, const std::vector<U> & v2) 
    -> std::vector<decltype(v1[0]+v2[0])>
{
  std::vector<decltype(v1[0]+v2[0])> r;
  auto i=v1.begin();
  auto j=v2.begin();
  auto end = v1.end();
  for (;i!=end;++i, ++j) {
    r.push_back((*i)+(*j));
  }
  return r;
}
\end{lstlisting}
\end{frame}

\begin{frame}[t,fragile]{Rules}
  \begin{itemize}
    \item \cppkey{decltype}\cppid{(expr)} 
          gets the type of an expression without evaluating it.
    \item \pause If \cppid{expr} names an identifier or accesses a member without extra parenthesis,
          its type is the type of the expression.
\begin{lstlisting}
int a;
struct point { double x, y; };
point p;
decltype(a) a1; // int a1
decltype(p.x) x1; // double x1
\end{lstlisting}
    \item \pause If \cppid{expr} is an r-value reference, 
          the type is an r-value reference.
\begin{lstlisting}
int && f();
decltype(f()) x; // int && x
\end{lstlisting}
  \end{itemize}
\end{frame}

\begin{frame}[t,fragile]{Rules}
  \begin{itemize}
    \item If \cppid{expr} is an l-value, the type is an l-value reference.
\begin{lstlisting}
int a;
decltype(a) x1; // int x1
decltype((a)) x1; // int & x1
\end{lstlisting}
    \item \pause In any other case, the type is the type of the expression.
\begin{lstlisting}
float x;
double y;
decltype(x+y) z; // double z;
\end{lstlisting}
  \end{itemize}
\end{frame}

