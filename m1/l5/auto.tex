\subsection{Deducing values in templates}

\begin{frame}[t,fragile]{Simplifying generic values}
\begin{itemize}
  \item Providing the type only for specifying a value is a common pattern.

\begin{block}{Generalized value}
\begin{lstlisting}
template <std::regular T, T, value>
bool is_less_than(T x) {
  return x < value;
}
\end{lstlisting}
\end{block}

\end{itemize}
\end{frame}

\begin{frame}[t,fragile]{Deduced values}

\begin{itemize}
  \item It can be simplified with \cppkey{auto} syntax.
    \begin{itemize}
      \item \textmark{Reminder}: \cppkey{decltype()} can be used to obtain
            the type of NTTP.
      \item \textbad{Note}: Default template parameter to name the type.
    \end{itemize}

\begin{columns}[T]

\column{.4\textwidth}
\begin{block}{Generalized value}
\begin{lstlisting}
template <auto value>
bool is_less_than(decltype(value) x) {
  return x < value;
}
\end{lstlisting}
\end{block}

\pause
\column{.6\textwidth}
\begin{block}{Generalized value with named type}
\begin{lstlisting}
template <auto value, typename T = decltype(value)>
bool is_less_than(T x) {
  return x < value;
}
\end{lstlisting}
\end{block}

\end{columns}

\end{itemize}
\end{frame}
